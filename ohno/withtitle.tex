\documentclass[11pt]{article}
\usepackage[utf8]{inputenc}
\usepackage[T1]{fontenc}
\usepackage{amsmath}
\usepackage{amsfonts}
\usepackage{amssymb}
\usepackage[version=4]{mhchem}
\usepackage{stmaryrd}

\DeclareUnicodeCharacter{2192}{\ifmmode\rightarrow\else{$\rightarrow$}\fi}

\begin{document}
\section*{Real Problem Identified:-}
Any ordered pair of real numbers ( $\mathrm{a}, \mathrm{b}$ ) is known as a complex number.\\
It is denoted by $z$ or $w$ and given by $z=(a, b)=a+i b$

$$
\text { where } \begin{aligned}
& a=\operatorname{Re}(z) \\
& b=\operatorname{Im}(z)
\end{aligned}
$$

i → imaginary unit of complex number.\\
Note : $-\sqrt{-1}$ is denoted by i and is pronounced by iota.

$$
\begin{aligned}
& z=(2+3 i) \\
& w=(-1+2 i) \text { etc. }
\end{aligned}
$$

Thus, $C=\{(a, b): a, b \in R\}$

Every real number is a complex number : $x$ is a real number we can write $x$ as :\\
$\mathrm{x}=\mathrm{x}+0 \mathrm{i}$\\
Thus $\mathrm{R} \subset \mathrm{C}$ (Real number system is the subset of complex number system) Where $R$ is the set of all real numbers and $C$ is the set of all complex numbers.

Any two complex number $\mathrm{z}=(\mathrm{a}, \mathrm{b})$ and $\mathrm{w}=(\mathrm{c}, \mathrm{d})$ are said to be equal if

$$
\begin{aligned}
& a=c \\
& \text { and } b=d
\end{aligned}
$$

i.e., real parts and imaginary parts are equal.

$\mathrm{z}=(\mathrm{a}, \mathrm{b})$ and $\mathrm{w}=(\mathrm{c}, \mathrm{d})$ be any two complex numbers then

$$
\begin{aligned}
z+w & =(a, b)+(c, d) \\
& =(a+c, b+d)
\end{aligned}
$$

Example:-

$$
\begin{aligned}
\mathrm{z} & =(3,4) \\
\mathrm{w} & =(2,1) \\
\mathrm{z}+\mathrm{w} & =(3+2,4+1) \\
& =(5,5)
\end{aligned}
$$

If $\mathrm{z}=(\mathrm{a}, \mathrm{b})$ and $\mathrm{w}=(\mathrm{c}, \mathrm{d})$ be any two complex numbers then

$$
\begin{aligned}
z \cdot w & =(a, b) \cdot(c, d) \\
& =(a c-b d, a d+b c)
\end{aligned}
$$

Example:

$$
\begin{aligned}
\mathrm{z} & =(2,3) \\
\mathrm{w} & =(2,5) \\
\text { then } \mathrm{z} \cdot \mathrm{w} & =(2.3-3.5,2.5+2.3) \\
\vdots & =(-9,16)
\end{aligned}
$$

(4) Division :\\
let $\mathrm{z}=\mathrm{a}+\mathrm{ib}$\\
$\mathrm{w}=\mathrm{c}+\mathrm{id}$ be any two complex numbers

$$
\begin{aligned}
\frac{z}{w} & =\frac{a+i b}{c+i d}=\frac{(a+i b)(c-i d)}{(c+i d)(c-i d)} \\
& =\frac{(a c+b d)+i(b c-a d)}{c^{2}+d^{2}} \\
& =\left(\frac{a c+b d}{c^{2}+d^{2}}\right)+\left(\frac{b c-a d}{c^{2}+d^{2}}\right)
\end{aligned}
$$

; Where $z \neq 0$

\section*{Example:-}

\end{document}