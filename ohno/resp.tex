\documentclass[11pt]{article}
\usepackage[utf8]{inputenc}
\usepackage[T1]{fontenc}
\usepackage{amsmath}
\usepackage{amsfonts}
\usepackage{amssymb}
\usepackage[version=4]{mhchem}
\usepackage{stmaryrd}
\usepackage{multirow}

\DeclareUnicodeCharacter{2192}{\ifmmode\rightarrow\else{$\rightarrow$}\fi}

\begin{document}
\begin{itemize}
  \item Oxidation of complex organic compounds (carbohydrates, lipids, proteins) releasing energy with help of enzymes.
  \item It is exothermic, exergonic, catabolic and oxido-reductive process.
\end{itemize}

$$
\mathrm{C}_{6} \mathrm{H}_{12} \mathrm{O}_{6}+\mathrm{O}_{2} \rightarrow \mathrm{CO}_{2}+\mathrm{H}_{2} \mathrm{O}+686 \mathrm{Kcal}
$$

\begin{itemize}
  \item Order of preferable respiratory substrate is: Glucose, fat, protein
  \item Glucose is the first preferred substrate of respiration and when complex carbohydrate and fats are used as respiratory substrate, it is called floating respiration.
  \item When protein is used as the respiratory substrate, it is called protoplasmic respiration.
  \item Depending upon the oxygen required, respiration may be aerobic or anaerobic respiration.
\end{itemize}

\begin{center}
\begin{tabular}{|l|l|}
\hline
Aerobic respiration & Anaerobic respiration \\
\hline
Takes place in presence of oxygen & Takes place in absence of oxygen \\
\hline
Complete oxidation of glucose & Incomplete oxidation of glucose \\
\hline
Takes place in cytoplasm and mitochondria & Takes place in cytoplasm \\
\hline
Completes in glycolysis, Krebs cycle and terminal oxidation. & Completes in glycolysis and incomplete oxidation of pyruvic acid. \\
\hline
Production of large amount of energy i.e 36-38 ATP or 686 Kcal. & Production of large amount of energy i.e 2 ATP or 36-50 Kcal. \\
\hline
Occurs in most of animals and higher plants. & Occurs in lower organisms (yeast and bacteria) and temporarily in few higher organisms. \\
\hline
End products are inorganic forms (i.e water and $\mathrm{CO}_{2}$ ). & End products are organic forms (Lactic acid, alcohol, acetic acids etc.) or both organic and inorganic forms (i.e $\mathrm{CO}_{2}$ ). \\
\hline
Water is produced as end product. & Water is not produced as end product. \\
\hline
Production of large number of $\mathrm{CO}_{2} \cdot$ & Less or no production of $\mathrm{CO}_{2}$. \\
\hline
\end{tabular}
\end{center}

\begin{itemize}
  \item Anaerobic respiration, also called fermentation is the exergonic or exothermic process and the term first used by Pasteur.
  \item The enzyme responsible for fermentation is Zymase first extracted by Buchner from yeast.\\
Fermentation can be of following types on the base of end product;
\end{itemize}

\section*{Alcoholic fermentation}
\begin{itemize}
  \item Occurs when oxygen is completely absent
  \item Exhibited by bacteria and yeast
  \item Completes in 3 steps; glycolysis, decarboxylation and dehydrogenation Glucose $\boldsymbol{\rightarrow}$ Pyruvic acid $\boldsymbol{\rightarrow}$ Acetaldehyde $\boldsymbol{\rightarrow}$ Ethyl alcohol $+\mathrm{CO}_{2}+$ energy
\end{itemize}

\section*{Acetic acid fermentation}
\begin{itemize}
  \item Occurs when oxygen is completely absent
  \item Exhibited by bacteria called Acetobacter aceti
  \item Completes in 3 steps; glycolysis, decarboxylation and dehydrogenation Glucose $\boldsymbol{\rightarrow}$ Pyruvic acid $\boldsymbol{\rightarrow}$ Acetaldehyde $\boldsymbol{\rightarrow}$ Acetic Acid $\boldsymbol{+} \mathbf{C O}_{\mathbf{2}} \boldsymbol{+}$ energy
\end{itemize}

\section*{Lactic acid fermentation}
\begin{itemize}
  \item Occurs when oxygen is insufficient in a cell (i.e. hypoxia)
  \item Exhibited by muscle cells, endoparasitic worms
  \item Completes in 3 steps; glycolysis, decarboxylation and dehydrogenation Glucose → Pyruvic acid $\boldsymbol{\rightarrow}$ Lactic acid + energy
\end{itemize}

\section*{Aerobic respiration has four continuous processes:}
\begin{itemize}
  \item Glycolysis in cytosol/cytoplasm (EMP pathway),
  \item Oxidative decarboxylation in outer and Perimitochondrial space,
  \item Kreb cycle in mitochondrial matrix (Tricarboxylic acid cycle or TCA cycle) but in bacteria it occurs in cytosol and
  \item Oxidative phosphorylation in inner membrane of mitochondria (Terminal oxidation, ETS)
\end{itemize}

\section*{1. Glycolysis}
\begin{itemize}
  \item Carbohydrate to enter into glycolysis, the glucose and fructose derived from hydrolysis of starch, sucrose or fructans must first be converted to hexose phosphates.
  \item Six carbon compounds - Glucose is broken down into 3-carbon compound pyruvic acid with net release of 2 ATP and $2 \mathrm{NADH}_{2}$.
  \item Occurs in cytoplasm of cell
  \item Also called EMP pathway - named after scientists who demonstrated (German biochemist, Gustab G. Embden and Otto F. Meyerhof and Parnas)
  \item Glycolysis includes 10 enzymatic steps,
  \item Irreversible reactions: conversion of glucose into glucose 6 phosphate, conversion of fructose\\
6 phosphate to fructose 1, 6-diphosphate.
  \item Primary Rate determining enzymes: phosphofructokinase.
  \item Reaction that does not require enzyme is conversion of 1,3 biphosphoglyceraldehyde into 1, 3 diphosphoglycerate.
  \item Total production of 4 ATP in glycolysis, 2 ATP while converting 1,3 diphosphoglycerate to 3 phosphoglycerate and 2 ATP while converting 3 phosphoenol pyruvate to pyruvate.
\end{itemize}

\section*{Glycolysis EMP pathway}
\begin{itemize}
  \item Total production of ATP in glycolysis $=10$ ATP
  \item Net production of ATP in glycolysis $=8$ ATP
  \item Total gain of ATP in glycolysis = 4 ATP
  \item Net gain of ATP in glycolysis $=2$ ATP
  \item Net gain of ATP in glycolysis through ETS = 6 ATP
  \item In aerobic glycolysis, in addition to 2 ATP molecules, reducing power, 2 NADH2 enter into mitochondria where they undergo ETS releasing 4 or 6 molecules of ATP.
\end{itemize}

\section*{Test Yourself}
Q.The most preferable substrate for respiration is carbohydrate (or glucose ) as compare to protein and fat. The energy produced by 1 gm of carbohydrate is\\
A. 4 calories\\
B. 7 calories\\
C. 9 calories\\
D. 8 calories

\section*{2. Oxidative decarboxylation}
\begin{itemize}
  \item In mitochondria, pyruvic acid produced through glycolysis goes on oxidation and decarboxylation forming 2 Acetyl CO A.
  \item In the outer membrane of mitochondria, pyruvic acid is decarboxylated into acetic acid with release of one $\mathrm{CO}_{2}$.
  \item The acetic acid is decarboxylated and combines with co - enzyme A to form Acetyl CoA by reducing NAD into NADH $+\mathrm{H}+$ and both reactions are catalyzed by pyruvate dehydrogenase (PDH) complex enzymes.
  \item Acetyl CoA is the most common substrate that enters into Krebs cycle -
  \item Acetyl CoA links glycolysis and Krebs cycle.
  \item 2 molecules of $\mathrm{CO}_{2}$ are released during link reaction.
  \item 2NADH2 produced in link reaction enter into ETS to form 6 ATP $(2 \times 3=6)$.
\end{itemize}

\section*{3. Krebs cycle (citric acid cycle or tricarboxylic acid cycle or TCA cycle)}
\begin{itemize}
  \item It is the oxygen consuming process in which acetyl CoA is successfully decarboxylated and dehydrogenated into $\mathrm{CO}_{2}, \mathrm{H}_{2} \mathrm{O}, \mathrm{NADPH}_{2}$ and $\mathrm{FADH}_{2}$ to liberate ATP in matrix of mitochondria. The Krebs cycle has 10 step of enzymatic reaction and discovered by German biochemist : Sir Hans Adolph

  \item Krebs $(1935,1945)$ and Johnson (1937) .

  \item Krebs was awarded by novel prize in 1953 in medicine.

  \item The first metabolic product of Krebs cycle is citric acid which contain three carboxylic acid group $(-\mathrm{COOH})$, hence called TCA or tricarboxylic acid cycle.

  \item The reactions of TCA cycle starts by condensation of acetyl - CoA with oxaloacetic acid (OAA) catalyzed by the citrate synthetase producing citric acid.

  \item All the reaction takes place in mitochondrial matrix except reduction of FAD which occurs in inner membrane of mitochondria because succinate dehydrogenase present only in inner membrane.

  \item It produces intermediate metabolites (both anabolic and catabolic) so called amphibolic or Anapleurotic pathway.

  \item Only one 5 carbon compound is formed during Kreb's cycle i.e $\alpha$ ketoglutaric acid.

  \item All the reactions are reversible (except reaction 2, 6 and 7)

  \item One molecule of Pyruvic acid releases - 3 molecules of $\mathrm{CO}_{2}, 4$ molecules of $\mathrm{NADPH}_{2}, 2$ molecules of $\mathrm{FADH}_{2}$ and 1 molecule of GTP (ATP).

  \item The connecting link between respiration and protein synthesis is aketoglutaric acid.

  \item From substrate phosphorylation in each Krebs cycle one ATP is formed. ATP is directly formed from GTP.

  \item There are $\mathbf{2}$ decarboxylation, 1 substrate phosphorylation, 4 dehydrogenation and 5 oxidation reaction in Krebs cycle.

  \item One molecule of glucose gives two molecules of pyruvic acid and from two kreb's cycle, 24

  \item ATP are formed (22 ATP from ETS and 2 ATP from substrate phosphorylation).

  \item In the oxidation reaction step, the hydrogen acceptor is NAD but in case of oxidation of succinic acid, the hydrogen acceptor is FAD.

\end{itemize}

\section*{4. Oxidative phosphorylation or ETS or Terminal oxidation}
\begin{itemize}
  \item It is the last step of aerobic respiration, the reduced coenzymes produced during glycolysis and Krebs cycle get oxidized by molecular oxygen.
  \item Synthesis of ATP by oxidation and reduction of cytochrome cofactors and coenzymes (NADH + and FADH + in presence of ATP synthetase present in oxysome (elementary particles or Fo and F1 particles) of inner membrane of mitochondria and called as oxidative phosphorylation.
  \item All the enzymes required for ETS are found in the F1 particles of mitochondria.
  \item In the ETS, electrons flow from more electronegative carrier to more electropositive carrier
  \item Oxidation and reduction process occurs in FAD, UQ, Cyt. complexes.
  \item The electron transport chain in mitochondria consists of 4 multiprotein electron transport complexes:
\end{itemize}

\begin{enumerate}
  \item Complex I (NADH dehydrogenase)
  \item Complex II (Succinate dehydrogenase)
  \item Complex III (Dihydroubiquinone - UQH ${ }_{2}$ )
  \item Complex IV (Cyt c - oxygen oxidoreductase)
\end{enumerate}

\begin{itemize}
  \item Energy released in from ETS stored in the form of ATP and every molecule of $\mathrm{NADH}+\mathrm{H}^{+}$produces 3ATP, FADH $+\mathrm{H}^{+}$produces 2 ATP.
  \item Energy is generated by the transfer of protons across the membrane which can be used for chemical, osmotic, or mechanical work - proton motive force.
  \item Each molecule of ATP can store 7.6 (7.3 or 8.9) Kcal energy.
  \item The chemicals that inhibit electron flow are: cyanide, azides and carbon monoxide that form a complex with cytochorome a3 and stop further transfer of oxygen.
  \item The reduced coenzymes $\mathrm{NADH}_{2}$ should enter into inner membrane of mitochondria, there are two compartments that $\mathrm{NADH}_{2}$ produced from glycolysis present in cytoplasm and cannot cross the membrane of mitochondria but it produced from Krebs cycle present in mitochondria. So there are two\\
mechanisms for ATP production: malate-aspartate shuttle and Glycerol phosphate shuttle.
  \item In malate-aspartate shuttle: oxaloacetate gains two hydrogen atom from NADH2 to form malate and malate after reaching membrane reduces NAD to NADH2 in the presence of malate dehydrogenase. The $\mathrm{NADH}_{2}$ produces 3 ATP molecules.
  \item In glycerol phosphate shuttle: dihydroxyacetone phosphate accepts two hydrogen atoms from $\mathrm{NADH}_{2}$ to form glycerol-3 phosphate. After reach to the membrane, it reduces FAD into $\mathrm{FADH}_{2}$ in presence of glycerol dehydrogenase. The $\mathrm{FADH}_{2}$ produce only two ATP.
\end{itemize}

\begin{center}
\begin{tabular}{|l|l|l|l|l|}
\hline
\multicolumn{5}{|l|}{Total ATP formed in aerobic respiration} \\
\hline
Pathway & From NADH + $\mathbf{H}^{\mathbf{+}}$ & from FAD $+\mathrm{H}^{+}$ & ATP (substrate) & total ATP \\
\hline
\multirow[t]{2}{*}{1. Glycolysis (glucosepyruvic acid)} & 2(4) GA shuttle & \multirow[t]{2}{*}{O} & \multirow[t]{2}{*}{O} & 6 \\
\hline
 & or 6 MA shuttle &  &  & or 8 \\
\hline
2. oxidative decarboxylation (pyruvate-acetyl CoA) & 2 (6) & 0 & o & 6 \\
\hline
3. Krebs cycle & 6 (18) & 2 (4) & 2 (GTP) & 24 \\
\hline
4. total ATP & 28 or 30 & 4 & 4 & 36 or 38 \\
\hline
\end{tabular}
\end{center}

\begin{itemize}
  \item 32 or 34 from ETS and 4 from without ETS (Substrate respiration)
  \item 36 to 38 ATPs are formed by complete oxidation of glucose or hexose sugar or C compound during aerobic respiration.
  \item 1 ATP molecule stores about 7.3 kcal of energy so the total energy stored as ATP is: $38 \times 7.3=277.4 \mathrm{kcal}$. This means that only 277.4 kcal are captured in the form of ATP out of 686 kcal energy available from glucose.\\
The efficiency of respiration is calculated as the fraction of the energy captured in ATP compared to the total energy stored in glucose.\\
Efficiency of respiration $=\frac{\text { energy captured in ATP }}{\text { Total energy stored in gulcose }} \times \mathbf{1 0 0} \%$
\end{itemize}

\section*{Respiratory quotient (RQ)}
\begin{itemize}
  \item It is the ratio of number of molecule of carbondioxide evolved to the number of molecules of oxygen consumed during respiration.
  \item RQ is measured by Ganong's respirometer.
  \item RQ value depends upon nature of substrate; ( i.e carbohydrate $=1$, protein $<1$, fat $<1$ while organic acids like malic acid $=>1$ ).
  \item The RQ value for aerobic respiration is equivalent to 1 while the value is less than 1 for anaerobic module respiration.\\
The RQ values for different plant parts and substrates are given below:
\end{itemize}

\begin{enumerate}
  \item Leaves rich in carbohydrate -1
  \item Darkened shoots of Opuntia - 0.03
  \item Germinating starchy seeds -1
  \item Germinating linseed (high fat) - 0.64
  \item Germinating buckwheat (high protein) - 0.5
  \item Germinating peas -1.54-2.4
\end{enumerate}

\section*{Factor Responsible for Respiration}
\begin{itemize}
  \item Light: It has no direct role in respiration. However, stomata open in presence of light which indirectly helps in exchange of respiratory gases. Respiratory rates in leaves adapted to full sun (sun leaves) are generally higher than those of leaves of the same species adapted to shade.
  \item Oxygen: Oxygen is essential for aerobic respiration but higher concentration oxygen inhibits the respiration which is known as Pasteur's effect.
  \item Temperature: The optimum temperature for respiration is $27^{\circ} \mathrm{C}$ and with every $10^{\circ} \mathrm{C}$ rise of temperature from $0-30^{\circ} \mathrm{C}$, the rate of respiration increases by $2-$ 2.5 times. It is called temperature coefficient.
  \item If the plant transferred from water to salt solution, the rate of respiration increases called salt respiration. It is due to absorption of ions require metabolic energy.
  \item Any injury, wounds and infections increase the rate of respiration for healing meristematic activities are required.
  \item Rate of respiration decreases with increase in age.
  \item Chemicals like, cyanides, azides, rotenone, antimycin, amytal, DNP etc inhibit the respiration.
  \item Increasing the concentration of carbondioxide decreases the rate of respiration. During respiration approximately 113 Kcal energy release per molecule of oxygen absorbed.
  \item Hormones like auxin, gibberellins, cytokinin increase the rate of respiration, specially ethylene is climacteric hormone that accelerates the rate of respiration.
\end{itemize}

\end{document}