\documentclass[11pt]{article}
\usepackage[utf8]{inputenc}
\usepackage[T1]{fontenc}
\usepackage{amsmath}
\usepackage{amsfonts}
\usepackage{amssymb}
\usepackage[version=4]{mhchem}
\usepackage{stmaryrd}
\usepackage{graphicx}
\usepackage[export]{adjustbox}
\graphicspath{ {./images/} }
\usepackage{caption}

\DeclareUnicodeCharacter{2192}{\ifmmode\rightarrow\else{$\rightarrow$}\fi}
\DeclareUnicodeCharacter{27F6}{\ifmmode\longrightarrow\else{$\longrightarrow$}\fi}

\begin{document}
\captionsetup{singlelinecheck=false}
\begin{itemize}
  \item It is the physiological process by green plants manufacture complex organic molecules (glucose) is formed by reaction of atmospheric $\mathrm{CO}_{2}$ and $\mathrm{H}_{2} \mathrm{O}$ in presence of light and chlorophyll.
  \item $\mathbf{6 C O}_{\mathbf{2}}+\mathbf{1 2 H}_{\mathbf{2}} \mathbf{O} \rightarrow \mathbf{C 6 H}_{\mathbf{1 2}} \mathbf{O}_{\mathbf{6}}+\mathbf{6 H}_{\mathbf{2}} \mathbf{O}+\mathbf{6 O}_{\mathbf{2}} \uparrow$
  \item It is the process of converting radiant energy into the chemical energy.
  \item Photosynthesis is reductive-oxidative process in which $\mathrm{CO}_{2}$ is oxidized into glucose and $\mathrm{H}_{2} \mathrm{O}$ is reduced to $\mathrm{O}_{2}$.
  \item It is endergonic and anabolic pathway.
  \item It results in increase in the dry weight\\
\includegraphics[max width=\textwidth, center]{9ce4edcc-acab-417d-acbc-d3578afcde2c-01_675_1030_700_249}
  \item Cyanobacteria and higher plants exhibit oxygeneic photosynthesis while photosynthestic bacteria and some red algae perform anoxygenic photosynthesis.
  \item More than $90 \%$ global photosynthesis carried out by phytoplankton (aquatic plants).
  \item Plant uses only visible light spectrums to perform photosynthesis. The visible light lies in electromagnetic spectrum ranging from 330 to 760 nm . These light energy different colors that we can visualize.
  \item The shorter the wavelength of light energy, the greater the energy it contains.
\end{itemize}

\begin{table}[h]
\begin{center}
\captionsetup{labelformat=empty}
\caption{Table: Radiation of light with average energy}
\begin{tabular}{|l|l|l|}
\hline
Colour & wavelength range(nm) & Average Energy Range (kJ mol-1 photons) \\
\hline
Ultraviolet & 100-400 &  \\
\hline
UV-C & 100-280 & 471 \\
\hline
UV-B & 280-320 & 399 \\
\hline
\end{tabular}
\end{center}
\end{table}

\begin{center}
\begin{tabular}{|l|l|l|}
\hline
UV-A & 320-400 & 332 \\
\hline
Visible & 400-740 &  \\
\hline
Violet & 400-425 & 290 \\
\hline
Blue & 425-490 & 274 \\
\hline
Green & 490-550 & 230 \\
\hline
Yellow & 550-585 & 212 \\
\hline
\end{tabular}
\end{center}

\begin{center}
\begin{tabular}{|l|l|l|}
\hline
Orange & $585-640$ & 196 \\
\hline
Red & $640-700$ & 181 \\
\hline
Far-red & $700-740$ & 166 \\
\hline
Infrared & longer than 740 & 85 \\
\hline
\end{tabular}
\end{center}

\begin{itemize}
  \item Photosynthesis is maximum in blue and red region of light spectrum.
  \item The most effective light for photosynthesis is red (best action spectrum).
  \item The least absorbed light is green and is reflected.
  \item The most absorbable light for photosynthesis is blue (best absorption spectrum).
  \item The spectrum between 400 nm and 700 nm light is photosynthetically active radiation (PAR).
  \item Photosynthetic pigments absorb and impart the light energy.
  \item The pigment that contains protein as an integral part of the molecule is known as a chromoprotein.
  \item The protein portion of a chromoprotein molecule is called the apoprotein.
  \item Chlorophyll is the primarily photosynthetic pigment responsible for converting the harvested light energy into chemical energy.
  \item Chlorophyll is magnesium porphyrin molecule containing a porphyrin head and a long hydrocarbon, or phytol tail.
\end{itemize}

\section*{Some researchers Related to the Photosynthesis-}
\begin{itemize}
  \item Leaves are organs for food preparation → Marcello Malphighi (1671).
  \item Air and light as constitute of building of plant body → Stephan Hales (1727).
  \item Green parts of plant are necessary for photosynthesis → Dutrochet (1837).
  \item Green plants convert solar energy into chemical energy of organic matter → Julius Robert Von Mayer (1845).
  \item The main source of carbon for plant is $\mathrm{CO}_{2} \rightarrow$ Liebig (1845).
  \item Terminology "chlorophylls" of green leaves → Pelletier and Caventon (1886).
  \item Two step reaction of photosynthesis: light reaction (photochemical) and dark reaction (thermochemical) → F.E Blackman (1905).
  \item Two photochemical cycle in light reaction → Emerson and Arnold (1918)
  \item Demonstration of photolysis of water: O2 evolved from water → Roberts Hills (1937).
  \item In bacterial photosynthesis, CO 2 assimilation is not accompanied by O 2 evolution $\boldsymbol{\rightarrow}$ C.B Von Neil (1941).
  \item Termed "photorespiration" to denote the carbon fixation without generation of ATP → Decker and Tio (1959).
  \item Discovery of quantasome (photosynthetic unit that state about 230 chlorophyll molecules ⟶ Parks and Biggins (1961).
\end{itemize}

\section*{Structure of Chloroplast}
\begin{itemize}
  \item Chloroplast - double membrane bounded organelle
  \item Consists of membrane envelop (outer membrane and inner membrane)
  \item The space between these two membranes is called inter-membrane space called interplastidal space
  \item Inner membrane encloses stroma/ matrix and stroma lamellae.
  \item The stroma is like the cytoplasm of the chloroplast, and contains numerous soluble enzymes, in particular the enzymes involved in carbon fixation.
  \item An extensive internal membrane system inside the chloroplast called the thylakoid membrane contains photosynthetic pigments and the electron transport system.
  \item Number of disc shaped thylakoids get piled upon one another forming the stack called granum.
  \item Grana are interconnected with each other by membranes called stroma lamellae or fret membrane.
  \item Quantasome (functional unit of photosynthesis) consists of small group of photosynthetic pigments.
  \item Each quantasome has 280 photosynthetic pigments ( 230 chlorophyll and 50 carotenoids).
\end{itemize}

\begin{center}
\begin{tabular}{|l|l|l|}
\hline
Chlorophyll & Carotenoids & Phycobillins \\
\hline
Chlorophyll a $\left(\mathrm{C}_{55} \mathrm{H}_{72} \mathrm{O}_{5} \mathrm{~N}_{4} \mathrm{Mg}\right.$ ) Green plants and cyanobacteria & Carotene ( $\mathrm{C}_{40} \mathrm{H}_{56}$ ) Orange Lycopene - Red & Phycocyanin - Blue Cyanobacteria \\
\hline
Chlorophyll 'b' ( $\mathrm{C}_{55} \mathrm{H}_{70} \mathrm{O}_{6} \mathrm{~N}_{4} \mathrm{Mg}$ ) - Green algae and all higher plants & Xanthophyll ( $\mathrm{C}_{40} \mathrm{H}_{56} \mathrm{O}_{2}$ ) Yellow colour Vioxanthin, Fucoxanthin (Brown algae) and Lutein & Phycoerythrin - Red algae \\
\hline
\end{tabular}
\end{center}

\begin{center}
\begin{tabular}{|l|}
\hline
Chlorophyll 'c' - Dinoflagellates, Diatoms and brown algae \\
\hline
Chlorophyll 'd' Red algae \\
\hline
Chlorophyll 'e' - Xanthophyceae \\
\hline
\end{tabular}
\end{center}

Difference between chlorophyll a and chlorophyll b

\begin{center}
\begin{tabular}{|l|l|}
\hline
Chlorophyll a & Chlorophyll b \\
\hline
It is bluish green in pure state. & It is olive green in pure state. \\
\hline
The empirical formula of Chlorophyll b is $\mathrm{C}_{55} \mathrm{H}_{70} \mathrm{O}_{6} \mathrm{~N}_{4} \mathrm{Mg}$ & The empirical formula of Chlorophyll b is $\mathrm{C}_{55} \mathrm{H}_{70} \mathrm{O}_{6} \mathrm{~N}_{4} \mathrm{Mg}$ \\
\hline
It is primary pigment. & It is accessory pigment. \\
\hline
Its molecular weight is 893 & Its molecular weight is 907. \\
\hline
It has an methyl group attached to 3 rd carbon of $2^{\text {nd }}$ pyrrole ring. & It has an aldehyde group attached to 3 rd carbon of $2^{\text {nd }}$ pyrrole ring. \\
\hline
It is more soluble in methyl alcohol. & It is more soluble in methyl alcohol. \\
\hline
It absorbs more red wavelength than violet blue wavelength of light. & It absorbs more violet blue wavelength than red wavelength of light. \\
\hline
\end{tabular}
\end{center}

\begin{itemize}
  \item Chlorophyll-a and carotenes are universal pigments while rest other are accessory pigments.
  \item The molar ratio of carotene and xanthophylls in young green leaves is $\mathbf{2}: \mathbf{1}$
  \item Major function of carotenoids is to protect chlorophyll from photo-oxidation so also called shield pigment.
  \item Water soluble pigment found in red and blue-green algae is phycobillins whose structure contain only 4 pyrrole ring and lacks Mg and phytol tails.
  \item Photosystems - two types of photosystems are present.
\end{itemize}

\begin{center}
\begin{tabular}{|l|l|}
\hline
PS I & PS II \\
\hline
Reaction center is designated as $\mathrm{P}_{700}$. & Reaction center is designated as $\mathrm{P}_{680} \cdot$ \\
\hline
Absorb light wavelength of both shorter and longer than 680 nm & Absorb light wavelength of both shorter and equal to 680 nm \\
\hline
Primary $\mathbf{e}^{-}$accepter is FeS compound & Primary $\mathbf{e}^{-}$accepter is Quinone type \\
\hline
Participates in both cyclic and non-cyclic photophosphorylation. & Participates only in non-cyclic photophosphorylation. \\
\hline
\end{tabular}
\end{center}

\begin{center}
\begin{tabular}{|l|l|}
\hline
Linked in formation of both ATP and NADPH & Linked in formation of both ATP \\
\hline
Found in non appressed region of grana and stroma lamellae & Found in apressed region of grana \\
\hline
Do not involve in photolysis of water and $\mathrm{O}_{2}$ liberation & Involves in photolysis of water and $\mathrm{O}_{2}$ liberation \\
\hline
\end{tabular}
\end{center}

\section*{Mechanism of Photosynthesis}
Completes in 2 major steps; light reaction and dark reaction.

\section*{A. light reaction:}
\begin{itemize}
  \item also called Hill reaction or Photochemical reaction or light dependent reaction was demonstrated by Robert Hill and Bendall (1960)
  \item involves two 4 steps
  \item Absorption of light
  \item Activation and excitation of Chlorophyll-a
  \item photolysis of water
  \item photophosphorylation.
\end{itemize}

\section*{1. Absorption of light energy}
\begin{itemize}
  \item The incident light on green parts of plants totally not absorbed and utilized, only about $0.5-3 \%$ ( average 1\%) light is used in photosynthesis.
  \item Photosynthetic pigment absorbs light of red and violet to blue part of spectrum known as Soret bands.
  \item The range of spectrum in which the rate of photosynthesis is maximum called as absorption spectrum and the series of absorption spectrum is:\\
White>>Red>>Blue>>Violet>>Indigo>>Orange>>Yellow>>Green
\end{itemize}

\section*{Red drop effect:}
\begin{itemize}
  \item Sudden decrease in quantum yield (number of molecule of O2 evolved per photon of light absorbed) in far- red light (above 680 nm ) than red light is called red drop effect.
  \item Light of shorter wavelength is provided along with longer wavelength light, the photosynthetic yield will increase which is known as red drop enhancement.
  \item Emerson discovered that when the two beams of light; red and far red light (i.e light wavelength of 680 nm and 700 nm ) applied simultaneously, the rate of\\
photosynthesis was two to three times greater than the sum of the rates obtained when monochromatic light is separately given.
  \item Presence of two photosystem in the pigment explains the functional process of absorption of light
\end{itemize}

Quantum yield: number of $\mathrm{O}_{2}$ evolved per quantum of light absorbed in photosynthesis.

\begin{itemize}
  \item Eight quanta light and 2 molecule of water is required for evolution of single molecules of $\mathrm{O}_{2}$ releasing 4 hydrogen ions (protons) and 4 electrons. [ 4 molecules of water are splitted to release 1 molecule of $\mathrm{O}_{2}$ ]
  \item The main feature of photosystem-II is photolysis of water leads to releasing of oxygen and reduction of $\mathrm{NADPH}_{2}$.
\end{itemize}

\section*{2. Activation and excitation of Chlorophyll-a}
\begin{itemize}
  \item Light energy absorbed by accessory pigment is transferred to the Chl-a then after receiving of photon of light by photosystem, the antenna pigment (chlorophyll-a) get excited.
  \item The unstable chl-a molecules comes to the metastable triplet state the chlorophyll charged positively though oxidation with the release of electrons and the electrons taken by water molecules.
\end{itemize}

\section*{3. Photolysis of water}
\begin{itemize}
  \item The excited chlorophylls donates electrons to the water molecules leads to splitting of water releasing oxygen and it is the first chemical reaction for photosynthesis.
  \item Photosystem II is associated with the thylakoid membrane (lumen side of thylakoid) and able to extract electron from water.
\end{itemize}

\section*{4. Photophosphorylation}
\begin{itemize}
  \item Process of formation of ATPs by adding inorganic phosphate to ADP in the presence of light.
  \item Occurs with the help of proton motive force.
  \item Two pathways - Non cyclic and cyclic
  \item The electron along with hydrogen ion produced from water are either cycled back or
  \item consumed on reducing NADPH $+\mathrm{H}+$ and producing ATP from ADP, the entire process is
  \item called photophosphorylation or photosynthetic phosphorylation.
  \item Arnon (1956) discovered the process of photophosphorylation:
  \item $\mathrm{NADPH}_{2}$ (reduced Nicotinamide Adenine Dinucleotide Phosphate) and ATP (Adenosine Triphosphate) together called assimilatory power.
  \item One assimilatory power is equivalent to:
  \item $3 \mathrm{ATP}+2 \mathrm{NADPH}+2 \mathrm{H}+$ and for the formation of a molecule of glucose, 6 assimilatory powers are required ie, $18 \mathrm{ATP}+12 \mathrm{NADPH}+12 \mathrm{H}+$.
  \item Photophosphorylation is of two types:
  \item Non- cyclic or Z-scheme (Hill and Bendall 1960) and cyclic photophosphorylation (Frenkel 1953).
\end{itemize}

\begin{figure}[h]
\begin{center}
  \includegraphics[max width=\textwidth]{9ce4edcc-acab-417d-acbc-d3578afcde2c-07_1035_1337_451_95}
\captionsetup{labelformat=empty}
\caption{Non cyclic photophosphorylation}
\end{center}
\end{figure}

\begin{itemize}
  \item Each pair of electrons passing through noncyclic electron transport from water to NADP+ contributes six protons to the gradient-four from the Q-cycle plus two from water oxidation.
  \item For cyclic electron transport, the number of protons transferred per pair of electrons is four.
\end{itemize}

\begin{figure}[h]
\begin{center}
  \includegraphics[max width=\textwidth]{9ce4edcc-acab-417d-acbc-d3578afcde2c-08_897_1435_51_11}
\captionsetup{labelformat=empty}
\caption{Cyclic photophosphorylation}
\end{center}
\end{figure}

\begin{center}
\begin{tabular}{|l|l|}
\hline
\multicolumn{2}{|c|}{Comparison between cyclic and non-cyclic photophosphorylation} \\
\hline
Non-cyclic or Z-scheme & Cyclic \\
\hline
Takes place in grana of chloroplast and both photosystems i.e., PS-I and PS-II absorb light & Involves only photosystem I \\
\hline
Produce single ATP and 2NADPH2 & Produce two ATP without production of NADPH 2 \\
\hline
Photolysis of water takes place & No photolysis of water \\
\hline
First electron donor is splitting water. & Electron donor and electron acceptor is same photosystem i.e PS I. \\
\hline
Terminal electron acceptor is NADP & Terminal electron acceptor is PS-I \\
\hline
Linked with evolution of $\mathrm{O}_{2}$ & Not linked with evolution of $\mathrm{O}_{2}$. \\
\hline
Predominant in higher plants and algae. & Predominant in bacteria. \\
\hline
\end{tabular}
\end{center}

\section*{Notes:}
\begin{itemize}
  \item $\mathrm{PQ}=\mathrm{Plastoquinone}$ (copper containing protein and structurally similar with vitamin K .
  \item $\mathrm{PC}=$ Plastocyanin (copper containing mobile protein)
  \item $\mathrm{FD}=$ Ferredoxin (iron non-haeme containing protein)
  \item FRS = Ferredoxin reducing substances (iron-sulphur protein)
  \item Cytochrome $=$ Iron or copper or both containing protein (common for photosynthesis and respiration).
\end{itemize}

\section*{Chemiosmotic theory for ATP generation}
\begin{itemize}
  \item Proposed by Peter Mitchell
  \item Occurred due to proton motive force
  \item Accumulation of $\mathrm{H}+$ ions in thylakoid lumen by 3 ways; photolysis of water, Reduction of PQ into PQH2 that pumps H + ions across the membrane into lumen.
  \item Reduction of NADP+ to NADPH2
  \item Inhibitors:Triazine, triazinone, urea like CMU and DCMU, amides inhibits the flow $\mathrm{e}^{-}$from PS II to PQ.
  \item Uncouplers: Dinitrophenol - prevents the formation of proton gradient.
\end{itemize}

\section*{Test Yourself}
Q. CMU and DCMU kills the weeds by inhibiting\\
A. Non cyclic photo phosphorylation\\
B. Cyclic photo phosphorylation\\
C. Dark reaction\\
D. Both cyclic and photo phosphorylation

Hint: CMU and DCMU prevent electron from PS II to PQ\\
Q. Assimilatory power are produced during phosphorylation. How many energy currency are produced in 2 turns of non-cyclic photo phosphorylation?\\
A. 4\\
B. 2\\
C. 6\\
D. 1

\section*{B. Dark reaction - Blackman reaction}
\begin{itemize}
  \item Biosynthetic step / Independent of light
  \item Consume assimilatory powers to form hexose sugar
  \item Occurs in stroma of chloroplast
  \item Can occur in 3 different pathways, C3, C4 and C2 cycle
  \item $\mathrm{C}_{3}$ Cycle - Calvin Cycle
  \item Proposed by Melvin Calvin (1954)
  \item Experiment demonstrated in Chlorella pyrenoidosa by utilizing 14CO2 radio isotopes
  \item Common and universal pathway found in majority of plants ( 95\% )
  \item First stable compound formed is 3-PGA (3 carbon compound)
  \item First carbohydrate formed is PGAL (3 carbon compound)
  \item The Calvin cycle can be divided into four primary stages:\\
a. Carboxylation: Fixation of $\mathrm{CO}_{2}$ by ribulose bisphosphate (RuBP) in the presence of enzyme Rubisco, and results in the formation of 3 carbon compound (i.e.3PGA) as first stable compound.
  \item First step of reaction in which ribulose, 5-biphosphate (RuBP) and CO2 combine to form highly unstable 6-carbon compound 2-carboxy- 3- keto 1, 5-diphosphorobitol which immediately splits to produce two molecules of $3-\mathrm{PGA}$ ).
  \item RuBisCo is the most abundant protein in the world, accounting for approximately 50 percent of the soluble protein in most leaves.\\
\includegraphics[max width=\textwidth, center]{9ce4edcc-acab-417d-acbc-d3578afcde2c-10_1078_1115_958_206}\\
b. Reduction: Reduction of 3 PGA into 3 PGAL ( 3 phosphoglyceraldehyde) utilizing ATP and NADPH produced by photochemical reaction.
  \item 3 PGAL connecting compound between photosynthesis and respiration.\\
c. Glycolytic reversal: conversion of 3 PGAL into hexose sugar, glucose.\\
d. Regeneration: Consumption of additional ATP to convert some of the PGAL back into RuBP to AG-Botany module 62 continue the carboxylation of $\mathrm{CO}_{2}$.
\end{itemize}

Difference between $\mathrm{C}_{3}$ and $\mathrm{C}_{4}$ plants

\begin{center}
\begin{tabular}{|l|l|}
\hline
C3 plants & C4 plants \\
\hline
Monomorphic chloroplast(mesophyll) & Dimorphic (grana in mesophyll and stroma in bundle sheath \\
\hline
Primary $\mathrm{CO}_{2}$ acceptor is RuBP & Primary $\mathrm{CO}_{2}$ acceptor is PEP \\
\hline
Carboxylation in one time in the cycle & Carboxylation occurs twice. \\
\hline
Photorespiration occurs & Photorespiration is negligible \\
\hline
Transpiration is high & Transpiration is low \\
\hline
$\mathrm{CO}_{2}$ compensation point is 25 100 ppm & $\mathrm{CO}_{2}$ compensation point is $1-20 \mathrm{ppm}$ \\
\hline
Calvin cycle takes place in mesophyll cell & Calvin cycle takes place in bundle sheath cell \\
\hline
Photosynthetic efficiency is less & Photosynthetic efficiency is more \\
\hline
Only C3 cycle occurs. & C4 cycle occurs in Bundle sheath cell and C4 cycle occurs in mesophyll cell. \\
\hline
For example: Rice, Wheat, Soyabean, Sunflower etc. & For example: Maize, Sorghum, Millet, Chenopodium, Amranthus etc. \\
\hline
\end{tabular}
\end{center}

\section*{Test Yourself}
Mesophyll cell of $\mathrm{C}_{3}$ plants is different from mesophyll cell of $\mathrm{C}_{4}$ plants in having\\
A. Granal chloroplast\\
B. Agranal chloroplast\\
C. Rubisco\\
d. PEPCo

\section*{Crassulacean Acid Metabolism (CAM) plants:}
\begin{itemize}
  \item Succulent xerophytic plants that open stomata at night time.
  \item Exhibits $\mathrm{C}_{4}$ cycle in day time and $\mathrm{C}_{3}$ at night time.
  \item 1st stable product - Malic acid
  \item For example: Cactus, Euphorbia, Pineapple etc.
  \item Main purpose - preventing excess transpiration in day time.
\end{itemize}

\section*{Photorespiration}
\begin{itemize}
  \item The wasteful pathway in which green plants respire (utilizing $\mathrm{O}_{2}$ and releasing $\mathrm{CO}_{2}$ ) in presence of light
\end{itemize}

\begin{figure}[h]
\begin{center}
  \includegraphics[max width=\textwidth]{9ce4edcc-acab-417d-acbc-d3578afcde2c-12_1411_1067_370_228}
\captionsetup{labelformat=empty}
\caption{Fig. Photorespiration cycle}
\end{center}
\end{figure}

\begin{itemize}
  \item The first respiratory substance is Glycolic acid (Glycolate) which is a 2 carbon compound.
  \item Exhibited by green plants especially C3 plants
  \item Different from mitochondrial respiration
  \item In chloroplast, enzymes RuBisCo perform oxygenase activity instead of carboxylation due to high concentration of oxygen and form phosphoglycolic acid and phosphoglyceric acid.
  \item Favoured in 3 Conditions; increased temperature, increased light intensity and increased $\mathrm{O}_{2}$ concentration
  \item Involves three organelles; chloroplast, peroxisome and mitochondria.
  \item About $50 \%$ of fixed energy waste due to photorespiration, Glycolic acid (glycolate) formed in chloroplast enter into peroxisome where it oxidized to glyoxylate or glyoxylic acid and $\mathrm{H}_{2} \mathrm{O}_{2}$.\\
Glyoxylate enter into mitochondria and convert into glycine by transamination reaction ana two molecule of glycine convert into single molecules of serine and one molecule of $\mathrm{CO}_{2}$ and $\mathrm{NH}_{3}$.
  \item Glycine and serine are amino acids involved in photorespiration.
  \item It is mostly pronounced in C3, absent in C4 and rare in CAM plants.
\end{itemize}

\section*{Test Yourself}
Q. Photosynthesis yield is higher in following plant\\
A. Sorghum\\
B. Rice\\
C. Wheat\\
D. Bryophyllum

Hint: Photosynthetic yield is higher in plants that doesn't exhibits photorespiration\\
Q. Vascular bundle is not surrounded by a sheath of cells containing agranal chloroplast that exhibit $\mathrm{C}_{3}$ cycle in\\
A. Sorghum\\
B. Sugarcane\\
C. Sunflower\\
D. Chenopodium\\
Q. During photorespiration, which organelles are involved in oxygen consuming reactions\\
A. Chloroplast\\
B. Peroxisome\\
C. Chloroplast and peroxisome\\
D. Chloroplast and peroxisome and mitochondria\\
Q. Photorespiration is called non-essential and wasteful process. However it occurs in $C_{3}$ plants for protecting the chloroplast from photooxidative damage releasing the most toxic compound.\\
A. $\mathrm{CO}_{2}$\\
B. $\mathrm{H}_{2} \mathrm{O}_{2}$\\
C. $\mathrm{NH}_{3}$\\
D. Serine

Bacterial Photosynthesis

\begin{itemize}
  \item Also called anoxygenic photosynthesis, first process explained by Van Neil (1950).
  \item Purple sulphur (Chromatium), green sulphur bacteria (Chlorobium) and nonsulphur bacteria (Rhodospirillum) can synthesize food by utilizing light without evolving oxygen.
  \item Bacterial chlorophyll which have $\mathbf{2} \mathrm{H}$ atoms more than chlorophyll-a, and can absorbs light (or infrared) of poor energy wavelength so cannot split water to evolve oxygen.
\end{itemize}

$$
6 \mathrm{CO}_{2}+12 \mathrm{H}_{2} \mathrm{~S} \rightarrow \mathrm{C}_{6} \mathrm{H}_{12} \mathrm{O}_{6}+6 \mathrm{H}_{2} \mathrm{O}+12 \mathrm{~S} \uparrow
$$

\section*{Bacterial photosynthesis}
\begin{itemize}
  \item Bacterial photosynthesis is anoxygenic.
  \item Photosynthesis in plants, algae and cyanobacteria is similar to bacterial photosynthesis in requirement for large amount of energy in the form of ATP (Adenosine Tri-phosphate) but different with respect to form of chemical reductants and resultant end products of photosynthesis.
  \item Water is chemical reductant of plants, algae and cyanobacteria.
  \item Inorganic compounds such as $\mathrm{H}_{2}$ and $\mathrm{H}_{2} \mathrm{~S}$ or and organic compounds lactate, succinate or malate are reductants of bacterial photosynthesis.
  \item Phototrophic bacteria using inorganic and organic chemical reductants are known as photolithotrophs and photo-organotrophs respectively.
\end{itemize}

\section*{Bacterial photosynthetic pigments:}
\begin{itemize}
  \item Bacteria have bacteriochlorophyll or bacterioviridin or chlorobium chlorophyll as photosynthetic pigment.
  \item Bacteriochlorophyll is the principle light harvesting pigment of photobacteria.
  \item It is present as a, b, c, d or e types.
  \item Bacteriochlorophyll absorb light in infrared region (wavelength of $725-1035 \mathrm{~nm}$ ).
  \item They are present in thylakoid that are scattered in the cytoplasm.
  \item Bacteria also contain carotenoids and other accessory pigments in addition to bacteriochlorophyll.
  \item Bacterial photosynthesis is based on cyclic photophosphorylation mechanism and only one pigment system (PS-I) is involved.
\end{itemize}

\section*{Factor Affecting Photosynthesis}
Concentration of $\mathrm{CO}_{2}$

\begin{itemize}
  \item According to the law of limiting factor by Blackman, the photosynthesis is affected by major three factors; light, temperature and concentration of $\mathrm{CO}_{2}$.
  \item The $\mathrm{CO}_{2}$ is most limiting factor for photosynthesis.
  \item Warburg effect: photorespiration is the inhibitory effect of increased oxygen concentration in C3 plants.
  \item At compensation point, photosynthesis and respiration are equal.
\end{itemize}

\section*{Temperature}
\begin{itemize}
  \item When $\mathrm{CO}_{2}$, light and other factors are not limiting, the rate of photosynthesis increases with a rise in temperature, over a range from $6^{\circ} \mathrm{C}$ to about $37^{\circ} \mathrm{C}$.
  \item Above this temperature, there is an abrupt fall in the rate and the tissue dies at $43^{\circ} \mathrm{C}$.
  \item High temperatures cause the inactivation of enzymes and therefore affect the enzymatically controlled 'dark' reactions of photosynthesis.
  \item Optimum temperature for photosynthesis is $25-30^{\circ} \mathrm{C}$.
\end{itemize}

\section*{Light}
\begin{itemize}
  \item The photosynthetically active region of the spectrum of light is at wavelengths from $400-700 \mathrm{~nm}$.
  \item Green light ( 550 nm ) plays an important role in photosynthesis. It induces physiological changes to plant morphogenesis eg. Making leaves wider and promote the opening of stomata.
  \item Light varies in intensity, quality and duration. A brief account on these three aspects is given as follows:
\end{itemize}

\section*{Intensity}
\begin{itemize}
  \item When $\mathrm{CO}_{2}$ and temperature are not limiting and light intensities are low, the rate of photosynthesis increases with an increase in its intensity.
  \item At a point saturation may be reached, when further increase in light intensity causes closure of stomata and reduce photosynthesis.\\
Factor Responsible for Respiration
\end{itemize}

Light effects on respiration during a subsequent dark period have been demonstrated but non-essential. For example, respiratory rates in leaves adapted to full sun (sun leaves) are generally higher than those of leaves of the same species adapted to shade.

The minimum concentration of oxygen for aerobic respiration is 1.9\% called as extinction point below that anaerobic respiration start and aerobic respiration stop.

The optimum temperature for respiration is and with every rise of temperature from the rate of respiration increases by 2-2.5 times.

If the plant transferred from water to salt solution, the rate of respiration increases called salt respiration. It is due to absorption of ions require metabolic energy.

Any injury, wounds and infections increase the rate of respiration for healing meristematic activities are required.

Rate of respiration decreases with increase in age.

Chemicals like, cyanides, azides, rotenone, antimycin, amytal, DNP etc inhibit the respiration.

Increasing the concentration of carbondioxide decreases the rate of respiration.During respiration approximately 113Kcal energy release per molecule of oxygen absorbed.

Hormones like auxin, gibberellins, cytokinin increase the rate of respiration, specially ethylene is climacteric hormone that accelerate the rate of respiration.

Respiratory quotient (RQ)

It is the ratio of molecule of carbondioxide evolves to the molecules of oxygen consumed during respiration and depends upon nature of substrate.

RQ is measured by Ganong’s respirometer.

The value of RQ vary upon the respiratory substrate. The RQ value is equal to 1 for complete oxidation of carbohydrate, less than 1 for protein and fat while more than 1 for organic acids like malic acid.

The RQ value for aerobic respiration is equivalent to 1 while the value is less than 1 for anaerobic respiration.

The RQ values for different plant parts and substrates are given below:

Leaves rich in carbohydrate -1

Darkened shoots of Opuntia -0.03

Germinating starchy seeds -1

Germinating linseed (high fat) -0.64

Germinating buckwheat (high protein) - 0.5

Germinating peas -1.54-2.4

\section*{a. Quality}
\begin{itemize}
  \item Most effective wavelengths differ with different plants.
  \item Plants show high photosynthesis in the blue and red light while red algae do so in green light and brown algae in blue light.
  \item Green light can penetrate deeper than other light.
  \item The blue-green algae have action spectrum peak in yellow or orange light.
\end{itemize}

\section*{b. Duration}
\begin{itemize}
  \item The rate of photosynthesis is more in the intermittent or discontinuous light than continuous light.
\end{itemize}

\section*{Oxygen}
\begin{itemize}
  \item Oxygen has been shown to inhibit photosynthesis in C 3 plants while C 4 plants show little effect.
  \item The rate of photosynthesis increases by $30-50 \%$ when the concentration of oxygen in air is reduced from 20\% to 0.5\%.
  \item Oxygen is inhibitory to photosynthesis because it would favour a more rapid respiratory rate utilizing common intermediates, thus reducing photosynthesis.
  \item Oxygen may compete with $\mathrm{CO}_{2}$ and proceeds photorespiration in $\mathrm{C}_{3}$ plants.
  \item Thirdly, $\mathrm{O}_{2}$ destroys the excited (triplet) state of chlorophyll and thus inhibits photosynthesis.
  \item An increase in oxygen concentration decreases photosynthesis and the phenomenon is called Warburg effect. [Reported by German scientist Warburg (1920) in Chlorella algae].
\end{itemize}

\section*{Water}
\begin{itemize}
  \item Water moisture in soil affects the water absorption and ascent of sap. It directly impacts the photosynthesis.
\end{itemize}

\section*{Minerals}
\begin{itemize}
  \item Presence of $\mathrm{Mn}^{++}$and CI- is essential for smooth operation of light reactions (Photolysis of water/ evolution of oxygen) $\mathrm{Mg}^{++}, \mathrm{Cu}^{++}$and $\mathrm{Fe}^{++}$ions are important for synthesis of chlorophyll.
\end{itemize}

\section*{Pollutants and Inhibitors}
\begin{itemize}
  \item The oxides of nitrogen and hydrocarbons present in smoke react to form peroxyacetyl nitrate (PAN) and ozone.
  \item PAN is known to inhibit Hill's reaction. Diquat and Paraquat (commonly called as Viologens) block the transfer of electrons between $Q$ and $P Q$ in $P S$ II.
  \item Other inhibitors of photosynthesis are monouron or CMU (Chlorophenyl dimethyl urea), diuron or DCMU (Dichlorophenyl dimethyl urea), bromocil and atrazine etc.
  \item DCMU and CMU, also known as competitive inhibitors, blocks the electron flow from PS II to PQ and inhibit further process.
  \item At low light intensities potassium cyanide appears to have no inhibiting effect on photosynthesis.
\end{itemize}

\end{document}