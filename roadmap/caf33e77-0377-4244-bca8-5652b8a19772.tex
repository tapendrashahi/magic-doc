\documentclass[11pt]{article}
\usepackage[utf8]{inputenc}
\usepackage[T1]{fontenc}
\usepackage{amsmath}
\usepackage{amsfonts}
\usepackage{amssymb}
\usepackage[version=4]{mhchem}
\usepackage{stmaryrd}
\usepackage{caption}
\usepackage{graphicx}
\usepackage[export]{adjustbox}
\graphicspath{ {./images/} }

\DeclareUnicodeCharacter{00D7}{\ifmmode\times\else{$\times$}\fi}
\DeclareUnicodeCharacter{21D2}{\ifmmode\Rightarrow\else{$\Rightarrow$}\fi}

\begin{document}
\captionsetup{singlelinecheck=false}
Idea Map Equivalent Weight of Formula

\begin{center}
\begin{tabular}{|l|l|l|}
\hline
Substance & Formula for Equivalent Weight (E) & Example \\
\hline
Acid & $E=\frac{\text { Molecular Mass }}{\text { Basicity }}$ & \( \begin{aligned} & \mathrm{H}_{2} \mathrm{SO}_{4} \rightarrow E=\frac{98}{2} \\ & =49 \mathrm{~g} / \mathrm{eq} \end{aligned} \) \\
\hline
Base & $E=\frac{\text { Molecular Mass }}{\text { Acidity }}$ & \( \begin{aligned} & \mathrm{Ca}(\mathrm{OH})_{2} \rightarrow E \\ & =\frac{74}{2}=37 \mathrm{~g} / \mathrm{eq} \end{aligned} \) \\
\hline
Salt & $E=\frac{\text { Molecular Mass }}{\text { Total charge }}$ & \( \begin{aligned} & \mathrm{Na}_{2} \mathrm{CO}_{3} \rightarrow E \\ & =\frac{106}{2}=53 \mathrm{~g} / \mathrm{eq} \end{aligned} \) \\
\hline
Redox Agent & \( \begin{aligned} & E \\ & =\frac{\text { Molecular Mass }}{\text { Change in O.N. in one molecule }} \end{aligned} \) & \( \begin{aligned} & \mathrm{KMnO}_{4} \rightarrow E \\ & =\frac{158}{5} \\ & =31.6 \mathrm{~g} / \mathrm{eq} \end{aligned} \) \\
\hline
Element & $E=\frac{\text { Atomic Mass }}{\text { Valency }}$ & \( \begin{aligned} A l \rightarrow E=\frac{27}{3} & \\ =9 \mathrm{~g} & / \mathrm{eq} \end{aligned} \) \\
\hline
\end{tabular}
\end{center}

\begin{table}[h]
\begin{center}
\captionsetup{labelformat=empty}
\caption{Important n -Factors for Redox Titrations}
\begin{tabular}{|l|l|l|}
\hline
Reagent & Reaction & nfactor \\
\hline
$\mathrm{KMnO}_{4}$ (Acidic medium) & \( \begin{aligned} \mathrm{MnO}_{4}^{-}+8 \mathrm{H}^{+}+5 \mathrm{e}^{-} & \\ & \rightarrow \mathrm{Mn}^{2+}+4 \mathrm{H}_{2} \mathrm{O} \end{aligned} \) & 5 \\
\hline
$\mathrm{KMnO}_{4}$ (Basic medium) & $\mathrm{MnO}_{4}^{-}+\mathrm{e}^{-} \rightarrow \mathrm{MnO}_{4}^{2-}$ & 1 \\
\hline
$\mathrm{K}_{2} \mathrm{Cr}_{2} \mathrm{O}_{7}$ & \( \begin{aligned} \mathrm{Cr}_{2} \mathrm{O}_{7}^{2-}+14 & \mathrm{H}^{+}+6 e^{-} \\ & \rightarrow 2 \mathrm{Cr}^{3+}+7 \mathrm{H}_{2} \mathrm{O} \end{aligned} \) & 6 \\
\hline
Oxalic Acid $\left(\mathrm{H}_{2} \mathrm{C}_{2} \mathrm{O}_{4}\right)$ & $\mathrm{C}_{2} \mathrm{O}_{4}^{2-} \rightarrow 2 \mathrm{CO}_{2}+2 e^{-}$ & 2 \\
\hline
Ferrous $\left(\mathrm{Fe}^{2+}\right) \rightarrow$ Ferric ( $\mathrm{Fe}^{3+}$ ) & $\mathrm{Fe}^{2+} \rightarrow \mathrm{Fe}^{3+}+e^{-}$ & 1 \\
\hline
\end{tabular}
\end{center}
\end{table}

\section*{Basic Titration Formula}
$$
\text { Molarity }(M)=\frac{\text { Number of moles of solute }}{\text { Volume of solution in liters (L) }}
$$

\begin{itemize}
  \item Unit: mol/L or M
\end{itemize}

$$
\text { Normality }=\frac{\text { Number of gram equivalents of solute }}{\text { Volume of solution in liters }(\mathrm{L})}
$$

\begin{itemize}
  \item Units: eq/L or N
\end{itemize}

\section*{Relation Between Molarity (M) and Normality (N)}
$$
N=M \times n \text { (where } n=\mathrm{n}-\text { factor } \text { ) }
$$

\begin{itemize}
  \item For Acid: $n=$ Basicity (No. of replaceable $\mathrm{H}^{+}$ions)
\end{itemize}

$$
\mathrm{H}_{2} \mathrm{SO}_{4} \rightarrow n=2
$$

\begin{itemize}
  \item For Bases: $n=$ Acidity (No. of replaceable $\mathrm{OH}^{-}$ions)
\end{itemize}

$$
\mathrm{NaOH} \rightarrow n=1
$$

\begin{itemize}
  \item For Salts: $n=$ Total charge on cation/anion
\end{itemize}

$$
\mathrm{Na}_{2} \mathrm{CO}_{3} \rightarrow n=2 \text { (since } \mathrm{CO}_{3}^{2-} \text { has } 2 \text { charge) }
$$

Dilution Formula ( $M_{1} V_{1}=M_{2} V_{2}$ )

$$
\begin{aligned}
& M_{1} V_{1}=M_{2} V_{2}(\text { For molarity }) \\
& N_{1} V_{1}=N_{2} V_{2}(\text { For normality })
\end{aligned}
$$

\begin{itemize}
  \item Used when a solution is diluted.
\end{itemize}

\section*{Titration Calculations (Acid-Base \& Redox)}
\section*{At Equivalence Point (Neutralization)}
$$
N_{\text {acid }} \times V_{\text {acid }}=N_{\text {base }} \times V_{\text {base }}
$$

\begin{itemize}
  \item For Monoprotic Acid ( HCl ) \& Monobasic Base ( NaOH ):
\end{itemize}

$$
M_{\text {acid }} \times V_{\text {acid }}=M_{\text {base }} \times V_{\text {base }}
$$

For Polyprotic Acids/Bases

$$
\left(\mathrm{H}_{2} \mathrm{SO}_{4}+\mathrm{NaOH}\right): M_{\mathrm{H}_{2} \mathrm{SO}_{4}} \times V_{\mathrm{H}_{2} \mathrm{SO}_{4}} \times 2=M_{\mathrm{NaOH}} \times V_{\mathrm{NaOH}}
$$

(Since $\mathrm{H}_{2} \mathrm{SO}_{4}$ is diprotic, $=2$ )

\section*{Redox Titrations (Using n-factor)}
$$
\text { Moles of solute }_{1} \times n_{1}=\text { Moles of solute } 2 \times n_{2}
$$

$\mathrm{KMnO}_{4}$ vs Oxalic Acid

$$
\begin{gathered}
M_{K M n o_{4}} \times V_{K M n O_{4}} \times 5=M_{\text {Oxalic Acid }} \times V_{\text {Oxalic Acid }} \times 2 \\
\left(\text { Since } n_{\mathrm{KMnO}_{4}}=5 \text { and } n_{\text {Oxalic Acid }}=2\right)
\end{gathered}
$$

Q. One mole of $\mathrm{Sn}^{2+}$ reduces ........ moles of $\mathrm{K}_{2} \mathrm{Cr}_{2} \mathrm{O}_{7}$\\
a. 3\\
b. 2\\
c. $1 / 3$\\
d. $1 / 2$\\
Q. 75 ml . of 0.2 NHCl will neutralize how many ml of $0.1 \mathrm{MBa}(\mathrm{OH})_{2}$ ?\\
a. 300 ml\\
b. 150 ml\\
c. 75 mld .38 ml

\section*{Percentage Purity of a Sample}
$$
\% \text { Purity }=\left(\frac{\text { Actual mass of pure substance }}{\text { Mass of impure sample }}\right) \times 100
$$

\section*{Equivalent weight}
\begin{itemize}
  \item No of the mass of a substance that combines with or displaces $\mathbf{1}$ gram of hydrogen ( $\mathbf{1 . 0 0 8} \mathbf{~ g ~ H}$ ), $\mathbf{8}$ grams of oxygen ( $\mathbf{0}$ ), or $\mathbf{3 5 . 5}$ grams of chlorine (Cl).
  \item It is crucial in stoichiometry, titrations, and redox reactions.
  \item Equivalent Weight $(E)=\frac{\text { Molar Mass (M) }}{n \text {-factor (n) }}$
  \item Units: grams per equivalent ( g/eq )\\
Q. 3.0 gm of metal is converted completely into its oxide. The weight of the oxide formed is 5.0 gm . The equivalent weight of the metal is\\
a. 18.00\\
b. 12.00\\
c. 24.00\\
d. 36.00
\end{itemize}

\section*{Calculation of $\mathbf{n}$-Factor}
It depends on the type of substance:\\
For Acids (Based on Replaceable $\mathbf{H}^{+}$Ions)

\begin{itemize}
  \item n-factor ( $\mathbf{n}$ ) = Basicity (No. of replaceable $\mathrm{H}^{+}$ions)
\end{itemize}

Examples:

\begin{itemize}
  \item $\mathbf{H C l} \rightarrow n=1$ (Monoprotic)
  \item $\mathbf{H}_{2} \mathbf{S O}_{4} \rightarrow n=2$ (Diprotic)
  \item $\mathbf{H}_{3} \mathbf{P O}_{4} \rightarrow n=3$ (Triprotic) eg. $\mathrm{H}_{2} \mathrm{SO}_{4}$
  \item Molar mass $=98 \mathrm{~g} / \mathrm{mol}$
  \item Basicity ( n ) $=2$
  \item $E=\frac{98}{2}=49 \mathrm{~g} / \mathrm{eq}$
\end{itemize}

\section*{For Bases (Based on Replaceable $\mathbf{O H}^{-}$Ions)}
\begin{itemize}
  \item n-factor (n) = Acidity (No. of replaceable $\mathrm{OH}^{-}$ions)
\end{itemize}

Examples:

$$
\begin{aligned}
& \mathrm{NaOH} \rightarrow n=1 \\
& \mathrm{Ca}(\mathrm{OH})_{2} \rightarrow n=2 \\
& \mathrm{Al}(\mathrm{OH})_{3} \rightarrow n=3
\end{aligned}
$$

eg. $\mathrm{Ca}(\mathrm{OH})_{2}$

\begin{itemize}
  \item Molar mass $=74 \mathrm{~g} / \mathrm{mol}$
  \item $\operatorname{Acidity}(\mathrm{n})=2$
\end{itemize}

$$
E=\frac{74}{2}=37 \mathrm{~g} / \mathrm{eq}
$$

\section*{For Salts (Based on Total Cation/Anion Charge)}
\begin{itemize}
  \item $n$-factor $(\mathrm{n})=$ Total positive or negative charge per formula unit
\end{itemize}

$$
\begin{aligned}
& -\mathrm{NaCl} \rightarrow n=1\left(\mathrm{Na}^{+}+\mathrm{Cl}^{-}\right) \\
& -\mathrm{Na}_{2} \mathrm{CO}_{3} \rightarrow n=2\left(\mathrm{CO}_{3}^{2-} \text { has } 2 \text { charge }\right) \\
& -\mathrm{Al}_{2}\left(\mathbf{S O}_{4}\right)_{3} \rightarrow n=6\left(\mathrm{Al}^{3+} \times 2=6\right)
\end{aligned}
$$

eg. $\mathrm{Na}_{2} \mathrm{CO}_{3}$

\begin{itemize}
  \item Molar mass $=106 \mathrm{~g} / \mathrm{mol}$
  \item Total charge $\left(\mathrm{CO}_{3}{ }^{2-}\right)=2$
\end{itemize}

$$
E=\frac{106}{2}=53 \mathrm{~g} / \mathrm{eq}
$$

For Redox Reactions (Based on Electron Transfer)

\begin{itemize}
  \item $\mathbf{n}$-factor $(\mathrm{n})=$ Change in oxidation number per molecule
\end{itemize}

Examples:

\begin{itemize}
  \item $\mathbf{K M n O}_{4}$ (in acidic medium) $\rightarrow \mathrm{MnO}_{4}^{-} \rightarrow \mathrm{Mn}^{2+}$ (Oxidation state change: $+7 \rightarrow+2 \Rightarrow n=5$ )
  \item $\mathrm{K}_{2} \mathrm{Cr}_{2} \mathbf{O}_{7} \rightarrow \mathrm{Cr}_{2} \mathrm{O}_{7}^{2-} \rightarrow 2 \mathrm{Cr}^{3+}$ (Oxidation state change: $+6 \rightarrow+3 \Rightarrow n=6$ )
  \item $\mathrm{FeSO}_{4}$ (Ferrous to Ferric) $\rightarrow \mathrm{Fe}^{2+} \rightarrow \mathrm{Fe}^{3+}$ (Oxidation state change: $+2 \rightarrow+3$
\end{itemize}

$$
\Rightarrow n=1)
$$

eg. $\mathrm{KMnO}_{4}$ (in acidic medium):

\begin{itemize}
  \item Molar mass $=158 \mathrm{~g} / \mathrm{mol}$
  \item Change in oxidation number $(\mathrm{n})=5$
  \item $E=\frac{158}{5}=31.6 \mathrm{~g} / \mathrm{eq}$\\
Q. The amount of $\mathrm{KMnO}_{4}$ required to prepare 100 ml of 0.1 N solution in alkaline medium is\\
a. 1.58 gm\\
b. 3.16 gm\\
c. 0.52 gm\\
d. 0.31 gm
\end{itemize}

Special Cases\\
For Elements

\begin{itemize}
  \item Equivalent weight $=\frac{\text { Atomic Mass }}{\text { Valency }}$
\end{itemize}

Example (Aluminium, Al):

\begin{itemize}
  \item Atomic mass $=27 \mathrm{~g} / \mathrm{mol}$
  \item Valency $=3$
  \item $E=\frac{27}{3}=9 \mathrm{~g} / \mathrm{eq}$
\end{itemize}

For Gases (At STP)

\begin{itemize}
  \item Equivalent volume $=\frac{\text { Molar Volume }(22.4 \mathrm{~L})}{\mathrm{n} \text {-factor }}$
  \item eg. $\mathrm{H}_{2}$
\end{itemize}

Molar volume $=22.4 \mathrm{~L}$\\
n-factor $=2\left(\quad\right.$ since $\left.H_{2} \rightarrow 2 H^{+}+2 e^{-}\right)$

$$
E_{\text {volume }}=\frac{22.4}{2}=11.2 \mathrm{~L} / \mathrm{eq}
$$

\section*{For Complex Compounds}
\begin{itemize}
  \item eg. $\mathrm{Na}_{2} \mathrm{H}_{2}$ EDTA
\end{itemize}

Molar mass $=372 \mathrm{~g} / \mathrm{mol}$\\
n -factor $=2$ (since it binds with 2 metal ions)\\
$E=\frac{372}{2}=186 \mathrm{~g} / \mathrm{eq}$

\section*{Applications of Equivalent Weight Titrations (Acid-Base \& Redox)}
\begin{itemize}
  \item Used in Normality (N) calculations: N $=\frac{\text { Weight (g) }}{\text { Equivalent Weight × Volume (L) }}$
\end{itemize}

\section*{Stoichiometric Calculations}
\begin{itemize}
  \item Helps balance chemical equations.
\end{itemize}

\section*{Electrochemistry}
\begin{itemize}
  \item Used in Faraday's laws of electrolysis:
\end{itemize}

$$
\text { Weight deposited }=\frac{\text { Current }(I) \times \text { Time }(t) \times \text { Equivalent Weight }}{96500}
$$

\section*{Different Concentration Terms used in Volumetric Analysis Mass Percentage (w/w \%)}
\begin{itemize}
  \item Mass of solute per 100 grams of solution.
\end{itemize}

$$
\text { Mass } \%=\left(\frac{\text { Mass of solute }}{\text { Mass of solution }}\right) \times 100
$$

\begin{itemize}
  \item Unit: Percentage (\%)
  \item A $20 \%(\mathrm{w} / \mathrm{w}) \mathrm{NaCl}$ solution means 20 g NaCl in 100 g solution.
\end{itemize}

\section*{Volume Percentage ( $\mathbf{v} / \mathbf{v}$ \%)}
\begin{itemize}
  \item Volume of solute per 100 mL of solution.
\end{itemize}

$$
\text { Volume } \%=\left(\frac{\text { Volume of solute }}{\text { Volume of solution }}\right) \times 100
$$

\begin{itemize}
  \item Unit: Percentage (\%)
  \item A $5 \%(\mathrm{v} / \mathrm{v})$ ethanol solution means 5 mL ethanol in 100 mL solution.
\end{itemize}

\section*{Mass by Volume Percentage (w/v \%)}
\begin{itemize}
  \item Mass of solute (in grams) per 100 mL of solution.
\end{itemize}

$$
\mathrm{w} / \mathrm{v} \%=\left(\frac{\text { Mass of solute }(\mathrm{g})}{\text { Volume of solution }(\mathrm{mL})}\right) \times 100
$$

\begin{itemize}
  \item Unit: Percentage (\%)
  \item A $10 \%(\mathrm{w} / \mathrm{v})$ glucose solution means 10 g glucose in 100 mL solution.
\end{itemize}

\section*{Parts Per Million (ppm)}
\begin{itemize}
  \item Mass of solute per million parts of solution.
\end{itemize}

$$
\mathrm{ppm}=\left(\frac{\text { Mass of solute }}{\text { Mass of solution }}\right) \times 10^{6}
$$

\begin{itemize}
  \item Unit: ppm (dimensionless)
\end{itemize}

5 ppm lead in water means 5 g lead in $10^{6} \mathrm{~g}$ solution (or $1,000 \mathrm{~L}$ ) of water.\\
Q. A 500 g toothpaste sample has 0.5 g fluoride concentration, then concentration of fluoride in ppm is\\
a. 200\\
b. 300\\
c. 400\\[0pt]
d. 1000 [CEE-2021]

\section*{Parts Per Billion (ppb)}
\begin{itemize}
  \item Mass of solute per billion parts of solution.
\end{itemize}

$$
p p b=\left(\frac{\text { Mass of solute }}{\text { Mass of solution }}\right) \times 10^{9}
$$

\begin{itemize}
  \item Unit: ppb (dimensionless)
  \item 2 ppb arsenic means 2 g arsenic in $10^{9} \mathrm{~g}$ solution (or $10^{6} \mathrm{~L}$ ) of water.
\end{itemize}

Mole Fraction (X)

\begin{itemize}
  \item Ratio of moles of one component to the total moles of all components.
\end{itemize}

$$
X_{A}=\frac{n_{A}}{n_{A}+n_{B}+n_{C}+\cdots}
$$

\begin{itemize}
  \item Unit: Dimensionless (no units)
  \item If a solution has 2 moles of ethanol and 8 moles of water, the mole fraction of ethanol is:
\end{itemize}

$$
X_{\text {ethanol }}=\frac{2}{2+8}=0.2
$$

Molarity (M)

\begin{itemize}
  \item Moles of solute per liter of solution.
\end{itemize}

$$
\text { Molarity }(\mathrm{M})=\frac{\text { Moles of solute }}{\text { Volume of solution (L) }}
$$

\begin{itemize}
  \item Unit: mol/L or M
  \item A 0.5 M NaOH solution means 0.5 moles of NaOH in 1 L of solution.\\[0pt]
Q. Molarity of pure water is: [CEE 2022]\\
a. 55.55 M\\
b. 5.55 M\\
c. 14 M\\
d. 555 M\\
Q. Molarity of $\mathbf{3 8} \% \mathbf{H C l}$ ( $\mathbf{1 . 1 9 ~ g m} \mathbf{~ m o l e}^{-\mathbf{1}}$ ) is\\
a. 11.38\\
b. 12.38 c .1 .238\\[0pt]
d. 10.38[CEE 2081]
\end{itemize}

Molality (m)

\begin{itemize}
  \item Moles of solute per kilogram of solvent.
\end{itemize}

$$
\text { Molality }(\mathrm{m})=\frac{\text { Moles of solute }}{\text { Mass of solvent }(\mathrm{kg})}
$$

\begin{itemize}
  \item Unit: mol/kg or m
  \item A 1 m NaCl solution means 1 mole of NaCl dissolved in 1 kg of water.\\
Normality (N)
  \item Number of gram equivalents of solute per liter of solution.
\end{itemize}

$$
\operatorname{Normality}(\mathrm{N})=\frac{\text { Gram equivalents of solute }}{\text { Volume of solution }(\mathrm{L})}
$$

\begin{itemize}
  \item Unit: eq/L or N
  \item A $1 \mathrm{NHH}_{2} \mathrm{SO}_{4}$ solution means 1 gram equivalent ( 49 g ) of $\mathrm{H}_{2} \mathrm{SO}_{4}$ in 1 L of solution.
\end{itemize}

\section*{Comparison Between Molarity and Molality}
\begin{center}
\begin{tabular}{|l|l|l|}
\hline
Feature & Molarity (M) & Molality (m) \\
\hline
Definition & Moles of solute per liter of solution & Moles of solute per kg of solvent \\
\hline
Dependence on Temperature & Changes with temperature (volume changes) & Independent of temperature (mass remains constant) \\
\hline
Preferred for & General lab use & Colligative properties (boiling point elevation, freezing point depression) \\
\hline
\end{tabular}
\end{center}

\section*{Interconversion Between Concentration Terms}
\begin{itemize}
  \item Molarity to Molality:
\end{itemize}

$$
m=\frac{M}{\text { Density of solution }(\mathrm{kg} / \mathrm{L})-(M \times \text { Molar mass of solute }(\mathrm{kg} / \mathrm{mol}))}
$$

\begin{itemize}
  \item Mole Fraction to Molarity:
\end{itemize}

$$
M=\frac{X_{A} \times \text { Density }(\mathrm{g} / \mathrm{mL}) \times 1000}{\text { Molar mass of solute }(\mathrm{g} / \mathrm{mol})}
$$

\section*{Applications of Different Concentration Terms}
\begin{itemize}
  \item Molarity (M): Used in titrations, chemical reactions.
  \item Molality (m): Used in colligative properties (e.g., boiling point elevation).
  \item ppm/ppb : Used in environmental chemistry (pollutant levels).
  \item Mole Fraction (X) : Used in gas mixtures and Raoult's law.
\end{itemize}

Concentration Terms

\begin{center}
\begin{tabular}{|l|l|l|l|}
\hline
Term & Formula & Units & Temperature Dependence \\
\hline
$\left(\frac{W}{W}\right) \%$ & $\frac{\text { Mass solute }}{\text { Mass solution }} \times 100$ & $\%$ & No \\
\hline
$\left(\frac{V}{V}\right) \%$ & $\frac{\text { Vol solute }}{\text { Vol solution }} \times 100$ & $\%$ & Yes \\
\hline
\end{tabular}
\end{center}

\begin{center}
\begin{tabular}{|l|l|l|l|}
\hline
$\mathbf{w} / \mathbf{v} \%$ & Mass solute (g) Vol solution (mL) $\times 100$ & \% & Yes \\
\hline
ppm & $\frac{\text { Mass solute }}{\text { Mass solution }} \times 10^{6}$ & ppm & No \\
\hline
ppb & $\frac{\text { Mass solute }}{\text { Mass solution }} \times 10^{9}$ & ppb & No \\
\hline
Mole Fraction (X) & $\frac{n_{A}}{n_{A}+n_{B}}$ & unitles & No \\
\hline
Molarity (M) & $\left(\mathrm{n}_{\text {mole }}\right)_{\text {solute }} \mathrm{V}_{\mathrm{L}}$ & mol/L & Yes \\
\hline
Normality & Gram equivalents L solution & eq/L & Yes \\
\hline
Molatity & $\left(\mathrm{n}_{\text {mole }}\right)_{\text {solute }}$ Mass of Solvent (kg) & mol/kg & No \\
\hline
\end{tabular}
\end{center}

\section*{Strength of a Solution (g/L)}
Strength $=$ Molarity $(\mathrm{M}) \times$ Molar Mass $(\mathrm{g} / \mathrm{mol})$

\begin{itemize}
  \item $1 \mathrm{MH}_{2} \mathrm{SO}_{4} \rightarrow$ Strength $=1 \times 98=98 \mathrm{~g} / \mathrm{L}$
\end{itemize}

Back Titration (Residual Titration)

\begin{itemize}
  \item Used when direct titration is not possible.
\end{itemize}

Moles of excess reagent - Moles of back-titrated reagent $=$ Moles reacted with analyte eg.

\begin{itemize}
  \item Step 1: Add excess HCl to $\mathrm{CaCO}_{3}$.
  \item Step 2: Titrate leftover HCl with NaOH .
  \item Step 3: Calculate reacted HCl :
\end{itemize}

Moles of HCl reacted $=$ Moles of HCl added - Moles of NaOH used\\
Q. 22 gm of oxalic acid is present in the 500 ml of the solution ( $1.1 \mathrm{gm} / \mathrm{ml}$ ) then percentage ( $\mathrm{w} / \mathrm{w}$ ) of oxalic acid is\\
a. $2 \%$\\
b. $4 \%$\\
c. 6\%\\
d. $8 \%$\\
Q. The normality of 1\% solution of sulphuric acid is nearly.\\
a. 1.0 N\\
b. 0.1 Nc .0 .2 N\\
d. 1.5 N\\
Q. Amount of $\mathrm{H}_{2} \mathrm{SO}_{4}$ present in $\mathbf{5 0 0 ~ c c}$ of $1.5 \mathrm{NH}_{2} \mathrm{SO}_{4}$ is:\\
a. 4.9 g\\
b. 49 g\\
c. 24.8 g\\
d. 36.74 g

Key Points to Remember\\
For Neutralization Titrations\\
$N_{1} V_{1}=N_{2} V_{2}$\\
For Redox Titrations:\\
Use n -factor method: $M_{1} V_{1} n_{1}=M_{2} V_{2} n_{2}$\\
For Percentage Purity:\\
Compare experimental mass with theoretical mass.\\
For Back Titrations:

\begin{itemize}
  \item Subtract excess moles from total moles added.
\end{itemize}

Different Terms used in Volumetric Analysis

\begin{itemize}
  \item Standard solution: Solution whose Concentration or strength is known is called as standard solution. Primary standard substance/solution: The substance whose standard solution can be prepared directly by weighing is known as primary substance and its solution is known as primary standard solution. e.g. $\mathrm{K}_{2} \mathrm{Cr}_{2} \mathrm{O}_{7}, \mathrm{AgNO}_{3}$, oxalic acid crystal, $\mathrm{Na}_{2} \mathrm{CO}_{3}$ (anhy.), sodium oxalate, Mohr's salt, etc. are primary standard substance.
  \item Primary standard substances should be non-toxic, non-hygroscopic, non-deliquescent, available in pure form, invariant composition in solid and solution state and have high molecular and equivalent weight to minimize error during weighing.
  \item Secondary standard substance: Substance whose standard solution can't be prepared by direct weighing but by titrating with primary standard solution are secondary standard solution. They are unstable, impure and volatile deliquescent/hygroscopic. Eg. $\mathrm{KMnO}_{4}, \mathrm{NaOH}, \mathrm{HCl}, \mathrm{H}_{2} \mathrm{SO}_{4}, \mathrm{HNO}_{3}, \mathrm{FeSO}_{4}$.
  \item Tirration: Process to determine strength of unknown solution (titrate) with the help of stam solution (titrant).
  \item Titration error: difference between end point and equivalence point.
  \item The solution taken in burette ⇒ Titrant ⇒ Standard Solution
  \item The solution taken in flask ⇒ Titrand or Titrate or analyte ⇒ Whose concn is to be determined
  \item Acidimetry: Process in which strength of acid solution is determined volumetrically by titrating with standard alkali solution in presence of indicator.
  \item Alkalimetry: determines the strength of alkali solution against standard acid solution.
  \item Permanganate titration: $\mathrm{KMnO}_{4}$ act as self indicator due to the formation of $\mathrm{Mn}^{2+}$ ion which shows the pink colour.
  \item For volumetric estimation of reducing agent (e.g. ferrous salt, oxalic acid, etc.), $\mathrm{KMnO}_{4}$ is used as standard oxidizing agent.
  \item Acid solution of $\mathrm{KMnO}_{4}$ doesn't give any precipitate but neutral and alkaline solution give.
  \item Dichromate titration: Standard $\mathrm{K}_{2} \mathrm{Cr}_{2} \mathrm{O}_{7}$ is used for estimating iron ores in presence of HCl .
  \item Iodometric titration: Liberated $\mathrm{I}_{2}$ from KI by means of chemical reaction is titrated against hypo solution. Note: $\mathrm{Ag}^{+}, \mathrm{Cu}^{2+}, \mathrm{Fe}^{3+}$ give iodometric titration but $\mathrm{Pb}^{2+}$ doesn't.
\end{itemize}

Which of the following cannot give iodometric titrations?\\
(a) $\mathrm{Fe}^{3+}$\\
(b) $\mathrm{Cu}^{2+}$\\
(c) $\mathrm{Pb}^{2+}$\\
(d) $\mathrm{Ag}^{+}$

\begin{itemize}
  \item End point/ Equivalence point: Termination point of reaction in titration as indicated by indicators. Nous $\mathrm{KMnO}_{4}$ can be acidified only with dil. $\mathrm{H}_{2} \mathrm{SO}_{4}$ because with $\mathrm{HCl}, \mathrm{HCl}$ is oxidised
  \item With $\mathrm{HNO}_{3}$ : It acts as oxidising agent.
\end{itemize}

\section*{Bonus point}
\begin{itemize}
  \item The indicator used in the titration of iodine against sodium thiosulphate is starch.
  \item In titration of strong acid and weak base, indicator used is methyl orange. e.g. $\mathrm{Na}_{2} \mathrm{CO}_{3}$ and HCl
  \item The pink colour of phenolphthalein in alkaline medium is due to negative form.
  \item Phenolphthalein is not a good indicator for titrating ferrous sulphate against $\mathrm{KMnO}_{4}$.
  \item If we use phenolphthalein as an indicator in a titration of $\mathrm{Na}_{2} \mathrm{CO}_{3}$ with HCl , the usual result is that no visible change will occur.
  \item Methyl orange gives red colour in hydrochloric acid solution.
  \item The best indicator for titrating 0.1 N solution of $\mathrm{Na}_{2} \mathrm{CO}_{3}$ with 0.1 NHCL is methyl red.
  \item Starch can be used as an indicator for the detection of traces of iodine in aqueous solution.
  \item Mohr's salt is dissolved in dil. $\mathrm{H}_{2} \mathrm{SO}_{4}$ instead of distilled water to prevent cationic hydrolysis.
  \item Acidified potassium permanganate solution is decolourised by Mohr's salt.
  \item The pH indicators are either weak acids or weak bases.
  \item Strong acids are generally used as standard solutions in acid base titration because they can be used to titrate both strong and weak bases.
  \item The no. of equivalents of $\mathrm{Na}_{2} \mathrm{~S}_{2} \mathrm{O}_{3}$ required for the volumetric estimation of one equivalent of $\mathrm{Cu}^{2+}$ is 2 .
  \item If 0.53 gm of $\mathrm{Na}_{2} \mathrm{CO}_{3}$ is added to 10 ml of N acid, the resulting solution is neutral.
\end{itemize}

\section*{pH Titration Curve}
\begin{center}
\includegraphics[max width=\textwidth]{d8e28d91-150b-447d-bd45-cd5a1f3f5589-12_956_1237_1055_187}
\end{center}

Note: Some questions are intentionally repeated to make your practice more significant.


\end{document}