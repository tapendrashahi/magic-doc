\documentclass[11pt]{article}
\usepackage[utf8]{inputenc}
\usepackage[T1]{fontenc}
\usepackage{amsmath}
\usepackage{amsfonts}
\usepackage{amssymb}
\usepackage[version=4]{mhchem}
\usepackage{stmaryrd}
\usepackage{caption}

\DeclareUnicodeCharacter{2713}{\ifmmode\checkmark\else{$\checkmark$}\fi}

\begin{document}
\captionsetup{singlelinecheck=false}
\section*{Cardiovascular diseases}
\begin{itemize}
  \item Cardiovascular: Related to the heart\\
✓ Affect the structures or function of the heart: Primary cause of death in the US\\
✓ Involves many heart-/blood vessel-related defects
  \item Definition of aetiology: Cause, set of reasons, or manner of causation of a disease or condition
  \item Disorder: An illness that disrupts a normal physical or mental structure and or functions
  \item Health: A state of complete physical, mental, and social wellbeing
  \item Heart attack: A myocardial infarction caused by insufficient blood by the heart
  \item Infarct: A dead heart muscle due to a blockage within heart arteries
  \item Infarction: Death of tissue due to lack of blood supply
  \item Vasoconstriction: Decrease in the size of the lumen of the blood vessel
  \item Self-medication: Amelioration of psychological distress via substance use
  \item Cardiac arrest: A sudden \& life-threatening condition in which heart stops beating
  \item Causes of cardiac arrest\\
✓ Heart attack/Myocardial infarction: Heart muscle damage due to interrupted bloodflow, mostly from a blocked artery\\
✓ Arrhythmias: Irregular heartbeats\\
✓ Heart diseases (Coronary artery disease, cardiomyopathy, heart failure)\\
✓ Electrolyte imbalance $\left(\mathrm{K}^{+}, \mathrm{Ca}^{2+}, \mathrm{Mg}^{2+}\right)$\\
✓ Electric shock\\
✓ Drug overdose\\
✓ Severe blood loss
\end{itemize}

\section*{- Treatment of heart diseases}
✓ Heart valve surgery, or coronary artery bypass graft surgery\\[0pt]
✓ Healthy and balanced diet [Low levels of fats; Low levels of salt; Low levels of sugar; Optimum fibres, fruits, \& vegetables]

\section*{1. Abnormal heart rhythms or arrhythmias}
\begin{itemize}
  \item Irregular heartbeat: 2 types (Tachycardia \& Bradycardia)
  \item Tachycardia: Fast heartbeat [Rate $>100$ beats/minute]\\[0pt]
✓ Atrial fibrillation: A rapid uncoordinated heartbeat [Temporary or Permanent]\\
✓ Atrial flutter: A rapid but organised heartbeat\\[0pt]
✓ Supraventricular tachycardia: Heartbeat starting from ventricles [Heartbeat starts \& stops suddenly]\\[0pt]
✓ Ventricular fibrillation: Rapid, uncoordinated electrical signals beginning at ventricles [Lack of squeezing blood from ventricles]\\[0pt]
✓ Ventricular tachycardia: Rapid heartbeat starting with faulty electrical signals in ventricles [Ventricles not properly filled with blood]
  \item Bradycardia: Heartbeat $<60$ beats per minute [If physically fit, heartbeat less than 60 beats per minute is normal \& healthy]\\[0pt]
✓ Sick sinus syndrome: Slow, disrupt, or block of heartbeat signals [Due to scarring near sinus node]\\
✓ Conduction Block: Skipped, stopping, or slowed beats due to block of electrical pathways
\end{itemize}

\begin{enumerate}
  \setcounter{enumi}{1}
  \item Aorta disease and Marfan syndrome: An inherited condition
\end{enumerate}

\begin{itemize}
  \item Prevents development of connective tissues
  \item Affect aorta, Heart valves\\
✓ Aortic aneurysm: Weakened \& bulging aorta\\
✓ Aortic dilation: Enlarged aorta
  \item Mainly 2 types\\
✓ Aortic valve regurgitation: Leaky heart valve between left aorta \& left ventricle\\
✓ Mitral valve prolapsed: Loose valve between upper \& lower pumping chambers on left hearts
\end{itemize}

\begin{enumerate}
  \setcounter{enumi}{2}
  \item Congenital heart disease: A group of structural abnormalities in heart present from birth
\end{enumerate}

\begin{itemize}
  \item One of the most common birth defects in about $1 \%$ live births globally
  \item Causes are maternal, genetic, \& environmental factors
  \item Types of congenital heart disease: Mainly 5\\
✓ Atrial Septal Defect: A hole in the interatrial wall\\
✓ Ventricular Septal Defect: A hole in the interventricular wall\\
✓ Tetralogy of Fallot/Blue Baby Syndrome: A hole between interventricular wall
  \item Narrowing of pulmonary valve/artery (Stenosis)
  \item Thickening of right ventricle muscular wall; Aorta situating over both ventricles not just in the left ventricle\\
✓ Transposition of the Great Arteries: Reversed arteries (2 main) leaving heart\\
✓ Coarctation of the Aorta: Narrowing of the aorta
\end{itemize}

\begin{enumerate}
  \setcounter{enumi}{3}
  \item Coronary artery disease/Coronary heart disease/Ischaemic heart disease: Most common heart disease
\end{enumerate}

\begin{itemize}
  \item Leading cause of death globally
  \item Occurs when coronary arteries are narrowed/blocked due to plaque (fatty deposition) via atherosclerosis
  \item Decreases bloodflow to heart
  \item Symptoms include chest pain (angina), heart attack or heart failure
\end{itemize}

\begin{enumerate}
  \setcounter{enumi}{4}
  \item Atherosclerosis: Narrowed \& hardened arteries due to plaques along inner wall
\end{enumerate}

\begin{itemize}
  \item Plaques [Fats, Cholesterol, \& Calcium]
  \item Develops in childhood \& progresses with age [A gradual disease]
  \item Mechanisms: Endothelial damage, plaque formation, plaque growth, plaque rupture, \& blood clots\\
✓ Obesity\\
✓ Diabetes\\
✓ Smoking
  \item Aetiologies: Plaque buildup enhanced by high levels of low density lipoprotein (LDL) \& inability of high density lipoprotein (HDL) to clear cholesterol from blood
\end{itemize}

\begin{enumerate}
  \setcounter{enumi}{5}
  \item Deep vein thrombosis: A blood clot (Thrombus) formation in one of the deep veins (Mostly in legs)
\end{enumerate}

\begin{itemize}
  \item Breaking of clot freely in blood and collection into lungs, leading to pulmonary embolism
\end{itemize}

\begin{enumerate}
  \setcounter{enumi}{6}
  \item Cardiomyopathy: A group of diseases that affect heart muscle (Myocardium) [Cannot pump blood effectively]
\end{enumerate}

\begin{itemize}
  \item Dilated cardiomyopathy: Dilation of heart chambers (Mostly left ventricle)
  \item Hypertrophic cardiomyopathy: Abnormal thickening (Hypertrophy) of myocardium [Sudden cardiac death in young athletes]
  \item Restrictive cardiomyopathy: Stiffening of myocardium
  \item Arrhythmogenic right ventricular cardiomyopathy: Myocardium is replaced by fatty or fibrous tissues affecting mostly right ventricle
\end{itemize}

\begin{enumerate}
  \setcounter{enumi}{7}
  \item Rheumatic heart disease: Permanent damage to heart valves due to rheumatic fever (Inflammatory disease caused by Streptococcus bacterial infection)
\end{enumerate}

\begin{itemize}
  \item Mitral valve stenosis (Narrowing) or Mitral valve regurgitation (Leaking) \& Aortic valve stenosis or Aortic valve regurgitation
\end{itemize}

\begin{enumerate}
  \setcounter{enumi}{8}
  \item Stroke: Interrupted or reduced bloodflow to a part of brain leading to lack of oxygen \& nutrients
\end{enumerate}

\begin{itemize}
  \item Brain cells begin to die within minutes, leading to long-term damage/disability
  \item Symptoms: FAST [Face drooping, Arm weakness, Speech difficulty, \& Time to call emergency services]
  \item Mainly 2 types\\
✓ Ischaemic stroke: 87\% of all strokes
  \item Blockage of an artery in the brain due to a blood clot or other obstruction, or atherosclerosis [Plaque in artery]\\[0pt]
✓ Haemorrhagic stroke: Rupture of blood vessels in brain [Bleeding/Haemorrhage within or around brain tissue]
\end{itemize}

\section*{Respiratory disorders}
\begin{itemize}
  \item Respiratory or pulmonary disorders are the most common medical conditions in the world
  \item Millions of people in the world are suffered from respiratory disorders.
  \item Aetiology of respiratory disorders: Smoking, Infections, Genes
\end{itemize}

\begin{enumerate}
  \item Asthma: A chronic disease
\end{enumerate}

\begin{itemize}
  \item Includes inflammation \& narrowing of the lung's airways
  \item Characterised by cough, wheeze, \& difficulty in breathing
  \item Different underlying mechanisms of asthma from COPD
  \item Aetiologies: Air pollution with microparticles and nanoparticles\\[0pt]
✓ Allergens [Pollens, dust, dust mites, fungi]\\
✓ Infections like sinusitis, colds, \& the influenza viruses\\
✓ Irritants like perfumes or cleaning solutions\\
✓ Tobacco smoke
  \item Symptoms \& pathology: Airway blockage due to tightening of the airways; Airway irritability; Coughing, especially at night or in the morning; Inflammation of bronchial tubes following damage to lungs; Shortness of breath; Tightness, pain, or pressure in the chest; Wheezing
  \item Treatment: Inhaled corticosteroids like beclomethasone, budesonide, fluticasone; Leukotriene modifiers; Long-acting beta-agonists
\end{itemize}

\section*{2. Chronic obstructive pulmonary disease (COPD)}
\begin{itemize}
  \item A group of diseases that cause respiratory airflow blockage \& breathing-associated problems
  \item A common, preventable, \& treatable chronic lung disease
  \item Mainly includes emphysema \& chronic bronchitis
  \item Caused 3.23 million deaths in 2019 globally (> 80\% deaths in low- \& middle-income countries)
\end{itemize}

\section*{Aetiologies}
\begin{itemize}
  \item Asthma in childhood
  \item Environmental factors (Tobacco smoke, indoor air pollution, dust, fumes, \& chemicals)
  \item Individual factors (Improper growth during childhood \& genetics)
  \item Long-term exposure to harmful gases and particles
\end{itemize}

Prevention: Avoiding smoking; Vaccinating against pneumonia, influenza, \& coronaviruses

\section*{A. Chronic bronchitis}
\begin{itemize}
  \item More common in people with asthma or emphysema
  \item Symptoms and Pathology: Chest discomfort; Coughing up mucus; Fatigue; Longterm wet cough; Pain in muscles (Muscle aches); Nasal congestion; Shortness of breath (Dyspnoea); Sore throat; A highpitched, coarse whistling sound during breath (Wheezing)
\end{itemize}

\section*{B. Emphysema}
\begin{itemize}
  \item Pathology: Lung damage; Breaking of the inner walls of the air sacs; Shortness of breath
\end{itemize}

\section*{3. Acute bronchitis}
\begin{itemize}
  \item A contagious infection, leading to inflammation of the bronchial tubes
  \item Symptoms and pathology: Coughing for several weeks; Feeling cold; Light fever ( $100^{\circ} \mathrm{F}$ ); Muscle aches; Runny nose; Sneezing; Sore throat; Tiredness; Wheezing; Mainly lasts for less than 10 days
\end{itemize}

\begin{enumerate}
  \setcounter{enumi}{3}
  \item Cystic fibrosis: An inherited disorder that causes severe damage to the lungs, digestive system, and other body organs; Caused by genes
\end{enumerate}

\begin{itemize}
  \item Genetic mutation (Cystic fibrosis conductance regulator, CFTR): Affects $\mathrm{Cl}^{-}$\& water transport in cells
  \item Inheritance of defective genes from both parents: Due to autosomal recessive condition
  \item Symptoms \& Pathology: Formation of sticky \& thick mucus; No clearing of mucus out of the bronchus; Multiplication of bacteria \& fungi in mucus; Difficulty to breathe; Chronic cough, Wheezing due to airway constriction; Nasal polyps; Finger clubbing (Enlarged fingertips due to chronic low $\mathrm{O}_{2}$ )
\end{itemize}

\section*{5. Coronavirus Disease 2019 [COVID-19]}
\begin{itemize}
  \item Aetiologies: Severe Acute Respiratory Syndrome Coronavirus 2 [SARS-CoV-2 virus 2] (A novel coronavirus); Thought to be zoonotic \& originated from animals that transmitted to humans
  \item Modes of transmission: Airborne; Fomiteborne (Substances/materials contaminated by pathogens)\\
✓ Via respiratory droplets (Sneezing, coughing, or talking)\\
✓ Aerosols (Particles flown in air, in poorly ventilated spaces/rooms)\\
✓ Direct contact with infected individuals \& Surface contact (Touching surfaces with virus)
  \item Symptoms: Fever \& chills; Cough; Breath shortness; Fatigue; Loss of taste or smell; Bodyaches; Headache; Sore throat; Runny nose; Nausea; Diarrhoea
  \item Pathology: Respiratory Syndrome (Inflammation \& infection in lungs leading to pneumonia \& acute respiratory distress syndrome, ARDS); Multi-system involvement (Heart, kidneys, liver) due to blood clot formation \& inflammatory responses; Cytokine storm [Overrelease of IL-6, Il-1 $\boldsymbol{\beta}$, IL-2, \& Interferon-gamma (IFN- $\boldsymbol{\gamma}$ )].
\end{itemize}

\begin{table}[h]
\begin{center}
\captionsetup{labelformat=empty}
\caption{TABLE: Vaccines used against COVID-19}
\begin{tabular}{|l|l|l|l|}
\hline
Vaccine & Company & \begin{tabular}{l}
Key \\
Ingredients \\
\end{tabular} & \begin{tabular}{l}
Storage \\
Temperature \\
\end{tabular} \\
\hline
\end{tabular}
\end{center}
\end{table}

\begin{center}
\begin{tabular}{|l|l|l|l|}
\hline
PfizerBioNTec h & Pfizer \& BioNTech & mRNA, lipid nanoparticles & $-90^{\circ} \mathrm{C}$ to $-60^{\circ} \mathrm{C}$ (longterm) $2^{\circ} \mathrm{C}$ to $8^{\circ} \mathrm{C}$ for 1 month \\
\hline
Moderna & Moderna & mRNA, lipid nanoparticles & $-50^{\circ} \mathrm{C}$ to $-15^{\circ} \mathrm{C}$ (longterm) $2^{\circ} \mathrm{C}$ to $8^{\circ} \mathrm{C}$ for 30 days \\
\hline
AstraZen eca & AstraZeneca, Oxford Univ. & Adenovirus vector & $2^{\circ} \mathrm{C}$ to $8^{\circ} \mathrm{C}$ \\
\hline
\end{tabular}
\end{center}

\begin{center}
\begin{tabular}{|l|l|l|l|}
\hline
Johnson \& Johnson & Johnson \& Johnson & Adenovirus vector, spike protein & $2^{\circ} \mathrm{C}$ to $8^{\circ} \mathrm{C}$ \\
\hline
Sinopharm & China National Pharmaceutic al Group & Inactivated virus, aluminium hydroxide & $2^{\circ} \mathrm{C}$ to $8^{\circ} \mathrm{C}$ \\
\hline
Sinovac (CoronaVa c) & Sinovac Biotech & Inactivated virus, aluminium hydroxide & $2^{\circ} \mathrm{C}$ to $8^{\circ} \mathrm{C}$ \\
\hline
Sputnik V & Gamaleya Research Institute & Adenovirus vector & $-18^{\circ} \mathrm{C}$ for frozen form $2^{\circ} \mathrm{C}$ to $8^{\circ} \mathrm{C}$ for liquid form \\
\hline
\end{tabular}
\end{center}

\section*{Renal disorders}
Involve disease or illness related to the structure and functions of the kidney

Family history of kidney failure; High blood pressure; Inflammation of the area of the kidney (Interstitial nephritis); Inflammation of glomeruli (Glomerulonephritis); Kidney-damaging medications; Older age; Prolonged destruction of urinary tract; Recurrent kidney infection (Pyelonephritis); Type 1 or type 2 diabetes; Use of tobacco; Urine back to the kidneys (Vesicoureteral reflux)

\section*{Symptoms \& Pathology}
Difficulty concentration; Dry, scaly skin; Frequent urination, especially late at night; Kidney failure; Lack of healthy RBCs (Anaemia); Poor appetite; Lack of penis erection (Erectile dysfunction); Malnutrition; Metallic taste; Muscle cramping; High blood pressure; Persistent itching; Puffiness around the eyes in the morning; Sleeping trouble; State of tiredness (Fatigue); Swollen feet/ankles; Weak bones

\section*{1. Chronic Kidney Disease (CKD)}
Aetiologies: Diabetes (High sugar level); Hypertension (High blood pressure); Glomerulonephritis (Inflammation of glomerulus); Polycystic Kidney Disease (Genetic disorder of cysts in kidneys); Obstruction of urinary tract by kidney stones; Long-term medication by non-steroidal antiinflammatory drug (NSAID) (Ibuprofen, Aspirin) \& antibiotics

Symptoms: Fatigue; Swelling of legs, ankles, face, Dark-coloured urine; Breath shortness; Nausea \& vomiting; Itchy skin; High blood pressure Pathologies: Reduced kidney function; Heart disease; Bone disease (Due to imbalance of calcium \& phosphorus minerals); Anaemia (Due to reduced erythropoietin production); Electrolyte imbalance; End-stage Renal Disease (ESRD) (Complete kidney failure requiring dialysis or transplant)

\section*{2. Kidney stone/Stone/Renal calculi/Nephrolithiasis/Urolithiasis Types of kidney stone}
i. Calcuim oxalate $\left(\mathrm{CaC}_{2} \mathrm{O}_{4}\right)$ : Most common [Due to dehydration or high oxalate levels] $\mathrm{Ca}^{2+}+\mathrm{C}_{2} \mathrm{O}_{4}{ }^{2-} \rightarrow \mathrm{CaC}_{2} \mathrm{O}_{4}$\\
ii. Uric acid $\left(\mathrm{C}_{5} \mathrm{H}_{4} \mathrm{~N}_{4} \mathrm{O}_{3}\right)$ : Produced in acidic urine [Due to high purine (Red meat) intake] $\mathrm{C}_{5} \mathrm{H}_{4} \mathrm{~N}_{4} \mathrm{O}_{3} \rightarrow$ Uric Acid Crystals\\
iii. Calcium Phosphate $\left(\mathrm{Ca}_{3}\left(\mathrm{PO}_{4}\right)_{2}\right)$ : Less common [Due to hyperparathyroidism]

$$
\mathrm{Ca}^{2+}+\mathrm{PO}_{4}{ }^{3-} \rightarrow \mathrm{Ca}_{3}\left(\mathrm{PO}_{4}\right)_{2}
$$

iv. Struvite $\left(\mathrm{MgNH}_{4} \mathrm{PO}_{4} \cdot 6 \mathrm{H}_{2} \mathrm{O}\right)$ : Due to urinary tract infections [UTIs] caused by ammoniaproducing bacteria

$$
\mathrm{Mg}^{2+}+\mathrm{NH}^{4+}+\mathrm{PO}_{4}{ }^{3-} \rightarrow \mathrm{MgNH}_{4} \mathrm{PO}_{4} \cdot 6 \mathrm{H}_{2} \mathrm{O}
$$

v. Cystine $\left(\mathrm{C}_{6} \mathrm{H}_{14} \mathrm{~N}_{2} \mathrm{O}_{4} \mathrm{~S}_{2}\right)$ : Rare [Due to an autosomal recessive genetic disorder called cystinuria or cystine in urine]\\
Cystine $\left(\mathrm{C}_{6} \mathrm{H}_{14} \mathrm{~N}_{2} \mathrm{O}_{4} \mathrm{~S}_{2}\right) \rightarrow$ Cystine Crystals\\
Aetiologies of kidney stone: Dehydration (Due to low fluid intake); High oxalate-rich food (Spinach), salt, \& animal protein; Family history of kidney stone; Obesity (Due to high calcium \& uric acid levels); Chronic diseases of diabetes, high blood pressure, \& gout (Due to urate crystal accumulation in joint); Certain drugs (Diuretics, Antacids)

\begin{itemize}
  \item Diuretics: Promote urine production \& increase the excretion of water and salts from the body
  \item Antacids: Neutralise stomach acid to relieve heartburn, indigestion, \& acid reflux
\end{itemize}

Symptoms of kidney stone: Severe pain (Sudden, sharp pain or renal colic in back, side, or lower abdomen); Haematuria (Blood in urine, causing pink, red, or brown colour urine); Frequent micturation; Nausea \& vomiting; Painful urination\\
Pathologies of kidney stone: Formation of stones (Calcium, oxalate, or uric acid crystallise in kidneys); Obstruction of urinary tract; Hydronephrosis (Back-up urine in kidney, with swelling \& kidney damage)

\section*{3. Polycystic Kidney Disease (PKD)}
\begin{itemize}
  \item Aetiologies of PKD\\
✓ Genetic Mutation: Mutations in PKD1 or PKD2 genes\\
✓ Autosomal Dominant Inheritance: Inherited from one affected parent (Most common form)\\
✓ Autosomal Recessive Inheritance: Inherited from both parents (Rare form)
  \item Types of PKD\\
✓ Autosomal Dominant Polycystic Kidney Disease (ADPKD): More common; Symptoms often appear in adulthood.\\
✓ Autosomal Recessive Polycystic Kidney Disease (ARPKD): Rare; Symptoms appear in infancy or early childhood.
  \item Symptoms of PKD: Abdominal pain (Due to enlarged kidneys); High blood pressure; Haematuria (Blood in urine); Enlarged abdomen; Kidney failure
  \item Pathologies: Cyst formation (Fluid-filled cysts in tubules); Reduced kidney function
\end{itemize}

\section*{4. Urinary tract infection (UTI)}
\begin{itemize}
  \item Aetiologies of UTI\\
✓ Bacteria: Escherichia coli, Klebsiella pneumoniae, Proteus mirabilis, Enterococcus faecalis and Staphylococcus saprophyticus\\
✓ Sexual activity; Diabetes; Kidney stones; Use of catheters; \& Hormonal changes (Menopause \& Pregnancy)
  \item Symptoms of UTI: Frequent urination; Burning sensation; Back pain; Fever \& Chills; Lower abdominal pain; Cloudy or Bloody urine with strong odour
  \item Pathology of UTI: Pyelonephritis (Infection spread from bladder to kidney); Cystitis (Inflammation of Bladder Lining); Kidney damage
\end{itemize}

\section*{5. Glomerulonephritis}
\begin{itemize}
  \item Inflammation of glomeruli (Filtering units of kidney)
  \item Aetiologies of glomerulonephritis\\[0pt]
✓ Autoimmune disorders: Lupus [Affects skin, heart, lung, kidneys, blood cells, \& brain]\\[0pt]
✓ Infections: Viruses [Hepatitis B \& C; HIV; Varicella-zoster virus); Bacteria [(Poststreptococcal glomerulonephritis by Streptocoocus throat infection); Escherichia coli; Staphylococcus; Pneumococcus]; Parasites [Plasmodium spp.]\\
✓ IgA nephropathy: IgA in glomeruli\\
✓ Genetic factors: Alport syndrome, a form of glomerulonephritis\\
✓ Types: Mainly 4 types
  \item Acute Glomerulonephritis: Sudden onset [Caused by an infection]
  \item Chronic Glomerulonephritis: Develops over time, leads to chronic kidney disease
  \item Primary Glomerulonephritis: Occurs on its own without other systemic diseases
  \item Secondary Glomerulonephritis: Linked to systemic diseases like lupus or diabetes
  \item Symptoms of glomerulonephritis: Haematuria (Blood in urine); Proteinuria (Protein in urine); Oedema (Swelling in face, hands, feet, \& abdomen); Hypertension (High blood pressure); Reduced urine output (Oliguria); Fatigue \& weakness
  \item Pathology of glomerulonephritis: Glomerular inflammation; Thickening of glomerular basement membrane; Accumulation of antigens \& antibodies
\end{itemize}

\section*{Substance abuse}
\begin{itemize}
  \item A disease that influences the addicted person's nervous system \& behaviour
\end{itemize}

\section*{Categories of drugs}
\section*{a. Narcotic drugs}
\begin{itemize}
  \item Lead to temporary feelings of wellbeing; Freedom from anxiety; Generating normal sleep, \& relief from pain
  \item Excess dose of these drugs leads to a reduction in respiration rates \& cardiovascular activities
  \item Examples of narcotics: Oxycodone, Codeine, Fentanyl, Hydrocodone, Tramadol, Morphine or Heroin, or Extracts of raw Opium Latex
\end{itemize}

\section*{b. Stimulant or antidepressant drugs (Mood-Elevating drugs)}
\begin{itemize}
  \item Cocaine, Caffeine, Nicotine, Amphetamines, \& Methylenedioxymethamphetamine (MDMA)
  \item Cocaine is obtained from the leaves of the coca plant (Erythroxylum coca).
  \item Stimulants release noradrenaline.
  \item Amphetamine (CNS stimulant) is also known as a superman drug.
\end{itemize}

\section*{c. Depressants or sedatives}
\begin{itemize}
  \item Slow down the functions of the CNS
  \item Switch off the activity of the brain
  \item Examples: Alcohol, Cannabis, Ketamine, Opioids (Heroin, Morphine, \& Codeine)
  \item Opioids include codeine, heroin, methadone, \& oxycodone, which give a feeling of wellbeing or euphoria.
  \item Opium is obtained from unripe seedpods (Capsule) of the poppy plant (Papaver somniferum).
  \item Opium has an analgesic (Pain-killing) effect and decreases blood pressure and breathing rate.
  \item Brown sugar, opium, is chemically diacetyl-morphine hydrochloride.
  \item Morphine, codeine, \& heroin are opium derivatives.
  \item Codeine is used in cough syrup.
  \item Heroin is the most dangerous narcotic.
  \item Sedatives in higher dose cause deep sleep.
  \item Barbiturates are sedatives.
\end{itemize}

\section*{d. Hallucinogens}
\begin{itemize}
  \item Affect our senses and change how we see, hear, taste, smell, or feel things
  \item Examples: $d$-lysergic acid diethylamide (LSD) or lysergide, $N, N$ Dimethyltryptamine (DMT), mescaline, psilocybin (magic mushroom), and phencyclidine (PCP)
  \item Also known as psychedelics (Greek; Psyche: Soul/mind + Delein: To manifest) or psychotomimetics
  \item Generate ego death or psychic death
  \item Ego death/Psychic death: Complete loss of self-identity
  \item Psychedelics: Strong substances that alter the perception and mood
  \item LSD is derived from ergot fungus (Claviceps purpura) that stimulates the smooth muscle.
  \item LSD was first synthesised by Albert Hofmann, a Swiss chemist, in 1938 from lysergic acid.
\end{itemize}

\section*{Alcoholism (Alcohol abuse or Alcohol addiction)}
\begin{itemize}
  \item Prolonged use of alcohol or alcoholic products leading to dependence on the body
\end{itemize}

\section*{Effects of alcohol on individual and health}
\begin{itemize}
  \item Nervous system: Loss of memory, self-control, \& Decision; Brain shrinkage; Dementia (Loss of memory, thinking, problem-solving, and language); Peripheral neuropathy (Nerve damage in extremities, leading to pain, tingling, and weakness)
  \item Oral cavity: Oral cavity cancer
  \item Oesophagus: Oesophageal cancer (Carcinoma of oesophageal squamous cell) if alcohol is combined with smoking
  \item Stomach: Gastritis; Stomach ulcer; \& Cancer
  \item Colon: Colon cancer
  \item Pancreas: Acute \& chronic Pancreatitis (Inflammation of the pancreas)
  \item Liver: Fatty liver; Hepatitis (A disease of inflammation of hepatic parenchyma or hepatocytes), Hepatomegaly (Enlargement of liver beyond its normal size); Cirrhosis (Scarring/fibrosis of liver; End stage liver); Liver cancer (Hepatocellular carcinoma); Liver failure
  \item Heart: Cardiomyopathy (Arrhythmic condition of heart due to muscle damage); Increased blood pressure; \& heart attack
  \item Kidney: Renal disorders; Dehydration \& electrolyte imbalance; Chronic kidney disease
  \item Reproductive organs: Infertility in both genders; Foetal alcohol syndrome (Drinking during pregnancy leads to neurodevelopmental issues on baby); Defective menstrual cycle and ovulation and lack of sexual desire
  \item Breast: Breast cancer
  \item Impact on immune system: Immunosuppression (Weaken immune system); Increased inflammation (various organs)
  \item Effects on family: Growth of anger, egoism, torture, pressure, tension, and stress among family members; Impact on the development of babies; Loss of properties; Lack of love, care, \& concern for family members
  \item Effects on community or society: Increase in corruption, rapes, \& accidents; Impact on social repute and qualities
  \item Temporary effects: Nausea and vomiting; Mood Change; Diarrhoea; Vision problems; Drowsiness (Sleepiness); Criminal activities
  \item In chronic alcoholics, the gamma-glutamyl transferase (GGT) enzyme is enhanced.
  \item Alcoholics are deficient in vitamin B complex (mainly thiamine) that can cause Karsakoff's psychoses and Wernicke encephalopathy.
  \item Higher dose of alcohol causes gastritis.
  \item Liver converts alcohol to a more toxic substance called acetaldehyde.
  \item Alcohol is the most common sedating drug.
  \item Alcohol is an aphrodisiac (Increasing sexual desire or libidoness).
\end{itemize}

Note: Previously it was thought that sexual desire occurs via the release of adrenaline, but now it is evident that it occurs via the release of dopamine.

\begin{itemize}
  \item Alcohol increases sexual desire but decreases performance.
  \item Alcohol decreases the secretion of antidiuretic hormone.
  \item Alcohol decreases the activity of the brain and spinal cord and reduces anxiety and tensions.
  \item Alcohol is the most commonly abused substance in the world.
\end{itemize}

\section*{Smoking}
\begin{itemize}
  \item Regular and prolonged consumption of tobacco in the form of cigarettes or chewing tobacco, leading to dependency
  \item Chemicals present in tobacco: 4000 chemicals, including nicotine, carbon monoxide, polycyclic hydrocarbons (carcinogenic), and tar
  \item Numbers of carcinogens evidenced in tobacco: 60
  \item Tar acronym: Total Aerosol Residue [A backronym, name coined for serious or humorous purpose; May be a false or folk etymology]
  \item Tar: Resinous, burned particulate matter produced by burning tobacco
  \item Tar's effects: Blacken teeth; Damage gums; Desensitise taste buds; Damage mouth; Damages lungs; Surrounds cilia \& stops mucus expel, leading eventual collection of allergens \& mucus within alveoli
  \item Tar as a cause of lung illness: Lung cancer; Emphysema
  \item Nicotine: Main addictive substance in cigarettes
  \item Chemical nature of nicotine: A naturally occurring chemical compound ( $\mathrm{C}_{10} \mathrm{H}_{14} \mathrm{~N}_{2}$ ) [Produced in roots and finally accumulated on leaves of plants]
  \item Regular smokers addicted to nicotine: 80-90\% of people
  \item Stimulates the conduction of nerve impulses
  \item Stimulates the adrenal medulla to secrete the adrenaline hormone
  \item Increases the rate of heartbeat and blood pressure
  \item Increases constriction of blood vessels
  \item Mimics the effects of acetylcholine
  \item Relaxes muscles
  \item Effect of smoking: Cancer by Carcinogens; Bronchitis; Emphysema (Rupture of the alveoli wall); Asthma (Contraction of Walls of bronchioles; Difficult breathing); Heart diseases
\end{itemize}

\section*{AG PLUS DOSE}
\begin{itemize}
  \item Party drugs are a group of stimulants \& hallucinogens.
  \item Inhalants are used to breathe via nose or mouth.
\end{itemize}

Example: Volatile solvents (Glues, Petrol), Aerosol sprays (Spray paints, Fly sprays, \& Deodorants), Gases (Propane, Butane, Helium, \& Nitrous oxide), \& Nitrites

\begin{itemize}
  \item Psychotropic or mood-altering drugs include sedative and tranquillisers (Anxiolytic), opiate narcotics, stimulants, \& hallucinogens.
  \item Tranquilisers reduce tension and anxiety without inducing sleep.
  \item Tranquilisers include chlorpromazine, diazepam, \& phenothiazines.
  \item The flowers \& Fruits (Marijuana), Flowering (Female) Buds (Hashish), Buds, Leaves, Flowers (Bhang), \& Charas (Resin of live plant) are obtained from the hemp plant (Cannabis sativus).
  \item May 31 is the Anti-tobacco Day.
  \item Tobacco is the most commonly abused substance in Nepal.
  \item Nauru has the highest smoking rate ( $52.1 \%$ ), with $51.7 \%$ males \& 52.60\% females.
  \item Nepal falls in 22 nd top country with a $31.9 \%$ smoking rate ( $48.6 \%$ males \& 15.30\% females)
  \item 22.3\% of the global population use tobacco [ 36.7\% of all men \& 7.8\% of the world's women]
  \item The annual death from tobacco consumption is more than 8 million ( $>7$ million: Smokers +1.2 million: Nonsmokers).
  \item 1.3 billion people are tobacco-users.
  \item More than $80 \%$ of tobacco-users live in low-and middle-income countries.
  \item E-cigarettes mean Electronic nicotine delivery systems (ENDS) and electronic non-nicotine delivery systems (ENNDS).
  \item MPOWER [Monitoring tobacco use; Preventing tobacco use; Offering help to quit; Warning; Enforcing; Raising taxes] is a way to reduce the demand for tobacco launched in 2007 by the WHO.
\end{itemize}

\end{document}