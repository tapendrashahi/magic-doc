\documentclass[11pt]{article}
\usepackage[utf8]{inputenc}
\usepackage[T1]{fontenc}
\usepackage{amsmath}
\usepackage{amsfonts}
\usepackage{amssymb}
\usepackage[version=4]{mhchem}
\usepackage{stmaryrd}

\begin{document}
\section*{1. Permutation:-}
The arrangements of a number of objects taken some or all of them at a time are called permutations. The total number of permutations of n distinct things taking $\mathrm{r}(1 \leq \mathrm{r} \leq \mathrm{n})$ at a time is denoted by ${ }^{\mathrm{n}} \mathrm{P}_{r}$ or by $\mathrm{P}(\mathrm{n} . \mathrm{r})$.

\section*{Note:-}
The number of permutations of n distinct objects taken $r$ at a time is given by ${ }^{\mathrm{n}} \mathrm{P}_{r}$ or $\mathrm{P}(\mathrm{n}, \mathrm{r})=\frac{n!}{(n-r)!}$ where $\mathrm{r} \leq \mathrm{n}$

\section*{2) Factorial:-}
(i) $n!=n \cdot(n-1) \cdot(n-2) \ldots 3 \cdot 2 \cdot 1$

$$
\begin{aligned}
& 5!=1 \cdot 2 \cdot 3 \cdot 4 \cdot 5=120 \\
& 6!=1 \cdot 2 \cdot 3 \cdot 4 \cdot 5 \cdot 6=720
\end{aligned}
$$

(ii) $0!=1$\\
(iii) $n!=n(n-1)!$\\
$10!=10.9$ !\\
(iv) $1!=1$\\
3) Note:-\\
(i) ${ }^{n} \mathrm{P}_{0}=\frac{n!}{(n-0)!}=\frac{n!}{n!}=1$\\
(ii) ${ }^{n} \mathrm{P}_{\mathrm{n}}=\frac{n!}{(n-n)!}=\frac{n!}{0!}=n!$

$$
[\because 0!=1]
$$

(iii) ${ }^{n} \mathrm{P}_{1}=\frac{n!}{(n-1)!}=\frac{n \cdot(r-1)!}{(n-1)!}=n$

$$
[n!=n(n-1)!]
$$

(iv) ${ }^{n-1} \mathrm{P}_{\mathrm{r}}+\mathrm{r} .{ }^{n-1} \mathrm{P}_{\mathrm{r}-1}={ }^{n} \mathrm{P}_{\mathrm{r}}$\\
4) The number of permutation of $n$ objects taken all at a time when $p$ objects are of one kind, $q$ objects are of second kind and $r$ objects are of third kind is

$$
=\frac{n!}{p!q!r!}
$$

\section*{Example:}
The numbers of permutations of the letters of the word "MATHEMATICS" are\\
*a) $\frac{11!}{2!\cdot 2!\cdot 2!}$\\
b) $\frac{10!}{2!\cdot 2!\cdot 2!}$\\
c) $\frac{1!}{2!\cdot 2!}$\\
d) $\frac{12!}{2!\cdot 2!\cdot 2!}$

Solution:-\\
There are 11 letters in the word MATHEMATICS\\
i.e $\mathrm{n}=11$\\
$A$ is repeated 2 times. $p=2$\\
$M$ is repeated 2 times. $q=2$\\
$T$ is repeated 2 times, $r=2$\\
Total number of arrangements $=\frac{11!}{2!\cdot 2!2!}$

\section*{5) Circular permutations:-}
a) Number of circular permutations of $n$ different things taken all at a time is $(n-1)!$.

\section*{Example:}
In how many ways 8 people and a host be seated in a circular table of a party?\\
a) 540 ways\\
b) 9! ways\\
c) 8! ways\\
d) 120 ways

Total number of persons $=8+1=9$\\
Solution:\\
Number of circular arrangements $=(9-1)!$ way $=8!$ way

\section*{Example:}
In how many ways 6 gentlemen and 4 ladies be scated in a round table?\\
Solution: -\\
Total number of way $=(10-1)!=9$ ! ways\\
b) The number of circular permutations of $n$ different things when an anticlockwise circular permutation and its corresponding clockwise circular permutation are considered as same circular permutation, then the number of circular permutation is $\frac{(n-1)!}{2}$

\section*{Example:}
Number of ways in which 6 different beads be arranged in a necklace is\\
a) 60\\
b) 120\\
c) 720\\
d) 540

Solution:

$$
\begin{aligned}
\text { Total number of ways } & =\frac{(n-1)!}{2} \\
& =\frac{(6-1)!}{2}=\frac{5!}{2} \\
& =\frac{120}{2}=60 \mathrm{ways}
\end{aligned}
$$

\begin{enumerate}
  \setcounter{enumi}{5}
  \item Number of permutations of $\mathbf{n}$ distinct objects taken $r$ at a time when the repetition of objects is allowed is $\mathrm{n}^{r}$.
\end{enumerate}

\section*{Example:-}
In how many ways can 6 books be distributed among 5 students?\\
Solution:-\\
required number of ways

$$
\begin{gathered}
=5 \times 5 \times 5 \times 5 \times 5 \times 5 \\
=5^{6} \\
{\left[\text { Hints }:-\mathrm{n}^{r}=5^{6}\right]}
\end{gathered}
$$

\section*{Example: -}
A child has 5 pockets and 3 coins. In how many ways can he put the coins in his pokets?\\
Solution:-\\
required number of ways $=5 \times 5 \times 5=5^{3}=125$ ways

\section*{7) Combination:-}
The selections (groups) of a number of things taken some or all of them at a time are called combinations. In combination, the order is not considered. The total number of combinations of n distinct things taking $\mathrm{r}(\mathrm{I} \leq r \leq n)$ at a time is denoted by ${ }^{n} C_{r}$ or by $C(n, r)$ The number of combination of $n$ distinct objects taken $r$ at a time is given by

$$
{ }^{n} \mathrm{C}_{\mathrm{r}}=\frac{n!}{(n-r)!r!}
$$

\section*{Properties:}
(i) ${ }^{n} \mathrm{C}_{0}={ }^{n} \mathrm{C}_{\mathrm{n}}=1$\\
(ii) ${ }^{n} \mathrm{C}_{1}={ }^{n} \mathrm{C}_{\mathrm{n}-1}=\mathrm{n}$\\
(iii) ${ }^{n} C_{r}={ }^{n} C_{s} \Rightarrow r=s$ or $r+s=n$\\
(iv) ${ }^{n} C_{r}+{ }^{n} C_{r-1}={ }^{n+1} C_{r}$\\
(v) $\frac{{ }^{n} C_{r}}{{ }^{n} C_{r-1}}=\frac{n-r+1}{r}$\\
(vi) ${ }^{n} \mathrm{C}_{0}+{ }^{n} \mathrm{C}_{1}+{ }^{n} \mathrm{C}_{2}+\cdots-\cdots+\cdots+{ }^{n} \mathrm{C}_{n}=2^{n}$\\
(vii) Number of ways of selecting one or more objects out of $n$ objects is ( $2^{n}-1$ ).

\section*{8) Relationship: -}
\section*{Permutations and Combinations:}
Permutation is the arrangement of objects and combination is the selection of objects.

$$
\begin{aligned}
{ }^{n} P_{r} & =\frac{n!}{n-r!} \\
{ }^{n} C_{r} & =\frac{n!}{(n-r)!r!} \\
{ }^{n} C_{r} & ={ }^{n} P_{r} \cdot \frac{1}{r!} \\
{ }^{n} P_{r} & =r!\cdot{ }^{n} C_{r}
\end{aligned}
$$

\section*{Example: -}
If ${ }^{\mathrm{n}} \mathrm{P}_{\mathrm{r}}=336$ and ${ }^{\mathrm{n}} \mathrm{C}_{\mathrm{r}}=36$, then $\mathrm{r}=$\\
a) 3\\
b) 4\\
c) 5\\
d) 6

Solution:-\\
We have:-

$$
\begin{aligned}
& { }^{n} C_{r}={ }^{n} C_{r} \cdot r! \\
& 336=36 \cdot r! \\
& 6=r! \\
& 3!=r! \\
& r=3
\end{aligned}
$$

Example: -\\
If ${ }^{2 n} C_{r}={ }^{2 n} C_{r+2}$ then $r=$\\
a) $n$\\
b) ( $\mathrm{n}+1$ )\\
*c) $(\mathrm{n}-1)$\\
d) $(n+2)$

\section*{Solution:-}
$$
\begin{aligned}
& r+r+2=2 n \\
& 2 r=2 n-2 \\
& r=(n-1)
\end{aligned}\left[{ }^{n} C_{r}={ }^{n} C_{s} \Rightarrow r+s=n\right]
$$

\section*{Example: -}
There are 5 questions in each group and B. In how many ways a student can answer questions from Group A and 2 from Group B?\\
a) 120\\
b) 150\\
c) 80\\
*d) 100\\
Solution:-

$$
\begin{array}{rl}
\text { Group A(5Q) } & \text { Group B(5Q) } \\
3 & 2 \\
\text { Total number of ways } & ={ }^{5} \mathrm{C}_{3} \times{ }^{5} \mathrm{C}_{2} \\
& =10 \times 10 \\
& =100
\end{array}
$$

\section*{Example:-}
The value of ${ }^{13} \mathrm{C}_{9}-{ }^{12} \mathrm{C}_{8}=$\\
Solution:-

$$
\begin{aligned}
& \left.=1^{12} C_{9}+{ }^{12} C_{8}\right)-{ }^{12} C_{8} \\
& ={ }^{12} C_{9}
\end{aligned}
$$

\section*{Example:-}
There are 5 subjects in an examination. In how many ways may a student fail?\\
a) 16\\
b) 15\\
c) 32\\
*d) 31\\
Solution: -\\
Total number of ways $=$

$$
\begin{aligned}
& ={ }^{5} C_{1}+{ }^{5} C_{2}+{ }^{5} C_{3}+{ }^{5} C_{4}+{ }^{5} C_{5} \\
& =2^{5}-1 \\
& =31
\end{aligned}
$$

\section*{Example: -}
The number of straight lines that can be formed by joining 8 points of which 4 points are collinear is\\
a) 14\\
b) 23\\
c) 32\\
d) 58

Solution: -

$$
\begin{aligned}
\text { total number of ways } & ={ }^{8} \mathrm{C}_{2}-{ }^{4} \mathrm{C}_{2}+1 \\
& =\frac{8.7}{1.2}-\frac{4.3}{1.2}+1=28-6+1=23
\end{aligned}
$$

\begin{enumerate}
  \setcounter{enumi}{8}
  \item (i) The number of combinations of n distinct objects taken $r$ at a time in which $p$ particular things always occur is ${ }^{n-p} C_{r-p}$.\\
(ii) The number of combinations of n distinct objects taken $r$ at a time in which $p$ particular things never occur is ${ }^{n-p} C_{1}$.\\
Note:-
\end{enumerate}

\begin{itemize}
  \item Number of $\Delta^{s}$ that can be formed from $n$ points $={ }^{n} C_{3}$
  \item Number of diagonals in a polygon $=\frac{n(n-3)}{2}=\mathrm{C}(\mathrm{n}, 2)-\mathrm{n}$
\end{itemize}

\section*{10. Derangements:-}
If $n$ things are arranged in a row, the number of ways in which they can be dearranged so that one of them occupies its original place is

$$
n!\left(1-\frac{1}{1!}+\frac{1}{2!}-\frac{1}{3!}+\cdots-\cdots+(-1)^{n} \frac{1}{n!}\right)
$$

\section*{Example:-}
There are 4 letters for correct placement in 4 envelops. The number of ways of wrong placement is\\
Solution:

$$
\begin{aligned}
\text { Number of ways of wrong } & =4!\left(1-\frac{1}{1!}+\frac{1}{2!}-\frac{1}{3!}+\frac{1}{4!}\right) \\
& =24\left(1-1+\frac{1}{2}-\frac{1}{6}+\frac{1}{24}\right) \\
& =9
\end{aligned}
$$

\section*{Example: -}
If ${ }^{n} \mathrm{P}_{2}=30$ then $\mathrm{n}=$\\
a) 5\\
b) 4\\
c) 6\\
d) 3

\section*{Solution: -}
$$
\begin{aligned}
& \frac{n!}{(n-2)!}=30 \\
& \frac{n(n-1)(n-2)!}{(n-2)!}=30 \\
& n(n-1)=30 \\
& n^{2}-n-30=9 \\
& n^{2}-6 n+5 n-30=0 \\
& (n-6)(n+5)=0 \\
& \text { either } n=6 \text { or } n=5
\end{aligned}
$$

\section*{Alternative Method}
directly put $n=6$

$$
\begin{aligned}
{ }^{6} P_{2} & =\frac{6!}{(6-2)!} \\
& =\frac{6.5 \cdot 4}{4!} \\
& =30
\end{aligned}
$$

\section*{Example: -}
How many number plates of vehicles consisting of 4 different digits can be made out of integers I, 2, 3, 4, 5, 6?

\section*{Example:-}
How many 6 different digits telephone numbers can be formed from the integers $0,1,2 \cdots-\cdots-\cdots$, by using the integers at once, which begins with 35 ?

\section*{Solution:-}
First two places are occupied by 3 and 5 remaining 4 places can be filled by 0,1,2,4,6,7,9 in

$$
{ }^{8} P_{4}=8 \times 7 \times 6 \times 5=1680 \text { ways }
$$

\section*{Example:-}
Number of ways in which the numbers greater than 100 and less than 4000 can be formed from the integers $0,1,2,3,4$ when repetition is allowed is:\\
*a) 375\\
b) 170\\
c) 400\\
d) 250

Solution:-\\
Since the number formed should be greater than 100 and less than 4000 there are three choices for the $1^{\text {st }}$ place (i.e. 1.2,3). The repetition is allowed so, there are 5 choices for each remaining places.

$$
\begin{aligned}
\text { Total number of ways } & =3 \times 5 \times 5 \times 5 \\
& =375
\end{aligned}
$$

\section*{Example:-}
Six A's have to be placed in the squares of the figure such that each row contains at least one A. In how many ways can this be done?\\
a) 26\\
b) 24\\
c) 16\\
d) 20

Solution:-\\
Six A's can be put in 8 squares in ${ }^{8} \mathrm{C}_{6}$ ways $=28$ ways\\
Excluding the remaining two ways, the required number of ways $={ }^{8} \mathrm{C}_{6}-2$

$$
=28-2=26
$$

\section*{Example:-}
The number of rectangle that can be formed from the following figure is:\\
a) $5!\times 5$\\
b) ${ }^{4} C_{2} \times{ }^{4} C_{2}$\\
c) ${ }^{5} P_{2} \times{ }^{5} P_{2}$\\
*d) ${ }^{5} \mathrm{C}_{2} \times{ }^{5} \mathrm{C}_{2}$

\begin{center}
\begin{tabular}{|l|l|l|l|}
\hline
 &  &  &  \\
\hline
 &  &  &  \\
\hline
 &  &  &  \\
\hline
 &  &  &  \\
\hline
\end{tabular}
\end{center}

\section*{Solution:-}
To form a rectangle, 2 horizontal lines and two Vertical lines are to be chosen from five horizontal and 5 vertical lines i.e.

$$
\text { Total number of ways }={ }^{5} \mathrm{C}_{2} \times{ }^{5} \mathrm{C}_{2}
$$

\section*{Example:-}
Among 14 players, 5 are bowler. In how many ways a team of 11 can be formed with at least 4 bowlers?\\
a) 312\\
*b) 264\\
c) 420\\
d) 512

\begin{center}
\begin{tabular}{|l|l|l|l|}
\hline
\begin{tabular}{l}
Players \\
(14) \\
\end{tabular} & \begin{tabular}{l}
Bowlers \\
(5) \\
\end{tabular} & \begin{tabular}{l}
Non bowlers \\
(9) (others) \\
\end{tabular} & Selection \\
\hline
 & 4 & 7 & ${ }^{5} \mathrm{C}_{4} \times{ }^{9} \mathrm{C}_{7}=180$ \\
\hline
 & 5 & 6 & ${ }^{5} \mathrm{C}_{5} \times{ }^{9} \mathrm{C}_{6}=84$ \\
\hline
\end{tabular}
\end{center}

Total number of ways $=264$

\section*{Example:-}
The number of ways in which $n$ distinct objects can be put in two different boxes so that neither of the boxes is empty is:\\
a) $2^{n}$\\
b) $2^{n}-1$\\
c) $2^{n}+1$\\
*d) $2^{n}-2$

\section*{Solution:-}
Each object can be put either in the $1^{\text {st }}$ box or in the $2^{\text {nd }}$ box i.e. 2 ways. Total number of ways $=2 \times 2 \times 2 \times$ $\_\_\_\_$ $\times n$ ways $=2^{n}$\\
Neither of the boxes is empty i.e. if the $1^{\text {st }}$ box is empty, put all the objects in $2^{\text {nd }}$ box and vice-versa.\\
Required number of ways $=2^{n}-2$

\section*{Example:-}
There are 5 books of Mathematics, 3 books of Physics and 4 books of Chemistry. The number of ways in which they can be placed on a shelf so that the books of the same subjects are always together is:\\
a) $5!.4!.3$ !\\
b) $(3!)^{2} \times 5!\times 4$ !\\
c) $3.20!$\\
d) $(5!\times 4!\times 3!) /{ }^{13} \mathrm{C}_{4}$

\section*{Solution:-}
5 books of mathematics, 3 books of physics and 4 books of chemistry can be arranged in $5!, 3!$ and $4!$ ways respectively.\\
Since the books of same subjects are always together, they can be arranged on a self in 3 !

$$
\begin{aligned}
\text { Total number of ways } & =3!\times 5!\times 4!\times 3! \\
& =(3!)^{2} \times 5!\times 4!
\end{aligned}
$$

\section*{Example:-}
The number of ways in which a student can select one or more questions out of 12 each having an alternative is:\\
a) $3^{12}$\\
b) $3^{12}-1$\\
c) $12^{3}$\\
d) $2^{12}$

Solution:-\\
Each question can be solved in three ways:\\
(i) first alternative may be selected\\
(ii) second alternative may be selected\\
(iii) neither may be selected

Hence 12 questions can be selected in $3^{12}$ ways.\\
But when none of the questions are solved, total number of ways $=3^{12}-1$


\end{document}