\documentclass[11pt]{article}
\usepackage[utf8]{inputenc}
\usepackage[T1]{fontenc}
\usepackage{amsmath}
\usepackage{amsfonts}
\usepackage{amssymb}
\usepackage[version=4]{mhchem}
\usepackage{stmaryrd}

\begin{document}
\begin{itemize}
  \item Chemical substances synthesized in specific glands/ organs and transported to other parts where they perform physiological responses even in extremely low concentration.
  \item Sterling (1906) coined the term Hormone in animal and J. Sachs (1856) suspected the presence of hormone in plants.
  \item Phytohormones (plant hormones) not only promote growth but some also inhibit growth- hence called plant growth regulators (PGRs) rather than phytohormones.
  \item Several hormones may affect the same response and active at small concentrations (in nM or even pM range).
  \item The effect increases with concentration and then saturates; may even be inhibitory at high concentration.
\end{itemize}

\section*{Features of plant growth regulators:}
These are simple organic molecules having several chemical compositions. They are also described as phytohormones, plant growth regulators, or plant growth hormones.

\begin{itemize}
  \item They can promote as well as inhibit the growth and development of plant. They have pleotropic effect. For example cytokinin promotes Cell division, cell elongation and differentiation, flowering etc.
  \item Same function is performed by number of plant hormones. For example; Cell elongation and cell division is promoted by Auxin as well as Gibberellins.
\end{itemize}

\section*{Discovery of Plant Growth Regulators}
\begin{itemize}
  \item The discovery of plant growth regulators began with investigation of phototropism and geotropism done by Charles Darwin and his son, Francis Darwin.
  \item They observed the growth of coleoptiles of canary grass (Phalaris canariensis) towards the light.
  \item That diffusible auxin was isolated by F.W. Went.
  \item Later, many scientists discovered and isolated different plant growth regulators. Gibberellins or gibberellic acid was formerly found in uninfected rice seedlings and was reported by E. Kurosawa
  \item Murashige and Skoog (MS) discovered another growth-promoting substance named kinetin, which is now known as cytokinins.
  \item Mainly, there are five types of plant hormones, namely, Auxin, Gibberellins (GAs), Cytokinins, Abscisic acid (ABA) and ethylene.
\end{itemize}

\section*{Auxin}
\begin{itemize}
  \item Charles Darwin (1880) discovered the auxin as growth stimulating substance from canary grass (Phalaris canariensis).
  \item Auxin was the first plant hormone to be discovered, also called juvenile hormone
  \item F. W. Went (1928) performed Avena coleoptile curvature (oat plant) test (bioassay of auxin) and gave the term Auxin with the basipetal and polar movement of Auxin. It is isolated in crystalline form from human urine by Kogl and Haagen Smit and called as Auxin A.
  \item Auxin B- isolated from corn grain oil (1934).
  \item Heteroauxin - isolated from human urine
  \item Thimann isolated IAA from culture filterate of Rhizopus.
  \item Auxins are biosynthesized in meristematic tissue, especially in shoot apical meristem and root apical meristem.
  \item They are also produced in leaves, developing fruits and seeds.
  \item Precursor - Tryptophan (Trp), an amino acid and Zinc
  \item Auxin transport takes place in polar basipetalous manner through the xylem elements, mainly the xylem parenchyma and also reported to move in acropetal manner in roots through phloem.
  \item Natural auxins; Indole acetic acid (IAA), Indole butyric acid (IBA), Phenylacetic acid, 4-Chloroindole-3-acetic acid.
  \item Synthetic auxins; Napthalene acetic acid (NAA), 2, 4 dichlorophenoxy acetic acid ( $2,4 \mathrm{D}$ ) and $2,4,5$ trichlorophenoxy acetic acid ( $2,4,5 \mathrm{~T}$ ), Indole butyric acid (IBA). The mixture of 2, 4 D and 2,4, 5 T (Agent orange)- used in Vietnam War by USA) - caused defoliation.
\end{itemize}

\section*{Physiological effects of Auxin}
\begin{itemize}
  \item Cell elongation: increases with increase in concentration and then saturates that may inhibit growth at high concentration due to auxin induced ethylene synthesis. Elongation is initiated in just 10 minutes after the application and continues upto 20 hrs .
  \item Callus formation: Callus is undifferentiated mass of cells. The application of auxin tissue culture helps in formation of callus.
  \item Cell division in vascular tissue: Auxin is distributed acropetally and promote the cell division in vascular cambium. It also helps in differentiation of vascular tissue. Phototropins, the photoreceptors of blue light signalling pathway form a lateral gradient and cause lateral movement of auxins to shaded side from where it moves to elongation zone where it stimulates cell elongation and bending of the axis.
  \item Root initiation: Induction of adventitious rooting in stem cuttings but root growth is inhibited.
  \item Apical Dominance: application enhances the growth of shoot tips leads to the main shoot grow fast.
  \item Gravitropism is also caused by lateral re-distribution of auxin due to gravity.
  \item Application of IAA in cut part in grafting and cutting can bring root initiation. Generally, it induces the initiation of later root and adventitious root.
  \item Delaying of the onset of leaf abscission- Abscission layer made at the base of leaf or fruit or flower that triggers the process of senescence.
  \item Parthenocarpic fruit development (Can substitute for pollination requirement in certain fruits like straw berry). Fruits with their achenes removed can resume growth normally when supplied with auxins exogenously.
  \item Flowering: Generally, auxin inhibits flowering. NAA and 2, 4 D induce flowering in Litchi and Pineapple.
  \item Eradication of weeds: Synthetic auxins like 2,4-D; 2,4,5-T; MCPA and Dicamba at high concentration act as selective herbicide and cause death of plants possibly due to excessive cell proliferation, epinastic cell bending and other developmental abnormalities.
  \item Phenoxy acetic acid based auxins are ineffective in killing plants like grasses that have P450 type monoxygenase which helps them metabolise these compounds. Production of ethylene: Auxin helps in formation of ethylene that prohibits the lateral bud initiation.
  \item Maintains dormancy of potato tuber, rhizomes and corms. (Sprouting is prevented by methylester)
  \item Induces nodule formation in leguminous plants.
  \item Prevention of lodging: Application of auxin makes the stem of cereal plants strong enough so that they become less prone to lodging.
  \item Healing: Auxin induces and initiates the cell division in cells surrounding the wound. Traumatin hormone is produced at injured area of plant.
  \item Sex expression: It brings the femaleness.
\end{itemize}

\section*{Gibberellins}
\begin{itemize}
  \item Kurasawa (1926) discovered gibberellins from Bakanae disease (foolish seedling) in paddy plants caused by Gibberella fujikuroi (Previously called as Fusarium monoliforme)
  \item Yabuta and Sumiki (1938) isolated pure crystals of active compounds from fungal cultures, called Gibberellins A and B.
  \item Brian (1955) isolated pure sample of a single gibberellin and named as gibberellic acid and structure was given by Cross et al., (1961).
  \item They are terpenoid compounds consisting of four isoprene units, i.e. Diterpene acids each isoprene unit consists of 5-C arranged in characteristic Gibbane or Gibberellane ring with variable numbers of $-\mathrm{OH}, \mathrm{COOH},-\mathrm{CH} 3$ groups and that of double bonds.
  \item More than 125 different GAs exist and only few GA1, GA2, GA3, GA4, GA7 and GA9 are typically used. Among these, GA3 is most commonly used Gibberellin.
  \item Bioassay: Dwarf maize/ Dwarf pea test and Cereal endosperm test
  \item Precursor: Isopentyl pyrophosphate
  \item Synthesized in chloroplast, found in embryo, buds, young leaves, immature seeds etc. but absent in genetically dwarf varieties.
\end{itemize}

\section*{Physiological Effects}
\begin{itemize}
  \item Stem elongation: Causes internodal elongation even in genetically dwarf varieties and increase wall extensibility without acidification.
  \item Bolting of rosette plants: Rosette plants like cabbage have reduced intermodal region. Application of Gibberellins in rosette plants induces a rapid elongation in its internodal region. The process is called bolting.
  \item Seed germination: GAs synthesized in embryos are transported to aleurone layer where they cause induce a-amylase required for the hydrolysis of endosperm reserves and helps in germination.
  \item Break dormancy of bud, tubers and seeds in plants.
  \item Parthenocarpy: It is more effective in formation of parthenocarpic fruits, mainly in Guava, Pome, apple, tomato, pear, grapes, cherries, apples etc.
  \item It induces flowering in long day plants.
  \item Sex expression: In certain plants like maize GAs suppress formation of staminate (male) flowers in ears while in dicots like cucumber, hemp and spinach they enhance the formation of staminate flowers and inhibit pistillate(female) flowers.
  \item It helps in delaying the ripening of fruits and help in storage of fruits, especially in citrus plants.
  \item Can be used as substitutes of Vernalization
  \item Flowering: Induces and promotes flowering in long day plants but inhibits flowering in short day plants.
\end{itemize}

\section*{Cytokinins}
\begin{itemize}
  \item Cytokinins are Adenine derivatives containing either isoperene or other groups (Benzyl, Furfuryl, etc.) attached to carbon atom in N6 position of Adenine ring.
  \item In 1956, Miller and his colleagues reported the isolation and crystallization of a highly active substance, identified as the adenine derivative N6furfurylaminopurine, from autoclaved herring sperm DNA named the substance kinetin.
  \item In 1965, Skoog and his colleagues proposed the term cytokinin.
  \item Zeatin- a natural cytokinin which is extracted from Maize.
  \item Bioassay: Carrot phloem culture, Retention of Chlorophyll (Richmond's effect)
  \item Precursor: Isopentyl adenosine
  \item A major site of cytokinin biosynthesis in higher plants is the root.
  \item High cytokinin levels have been found in roots, especially the mitotically active root tip, and transported through the xylem but a signal from shoots is essential for such transport.
  \item Cytokinins are highly essential for the plant as the mutants that lack cytokinins are lethal.
\end{itemize}

\section*{Physiological Effects}
\begin{itemize}
  \item Cytokinin regulates cell division in shoots and roots.
  \item Promotes counteraction of apical dominance or lateral branching (shows antagonistic behabiour than that of auxins).
  \item Elevated levels of Cytokinin also causes abnormal growth like Witches' broom.
  \item Cytokinins delay leaf and organ senescence in intact as well as detached parts.
  \item Influence mobilization of nutrients from leaves to other parts.
  \item Treatment of etiolated (lack chlorophyll pigment) seedlings with cytokinins prior to illumination leads to formation of chloroplasts with more extensive grana and with more chlorophyll and photosynthetic enzymes indicating the role of cytokinins along with other (light, nutrients, etc) in chloroplast development.
  \item Promotes cell expansion in cotyledons and leaves in dicot plants with epigeal germination like mustard, cucumber, sunflower, etc.
  \item Cell enlargement: Expansion caused by increased cell wall extensibilty but without proton extrusion.
  \item Differentiation and morphogenesis: The proper ratio of auxin and cytokinin helps to differentiate tissue and form plant parts.
\end{itemize}

\section*{1.cytokinin > auxin - shoot formation}
2.cytokinin < auxin - root formation,\\
3.cytokinin = auxin - callus formation without differentiation.

\begin{itemize}
  \item Delay senescence: Spraying cytokinin on detached leaves makes them fresh for long time than the plucked leaves in controlled condition.
  \item It provides resistance to pathogen and extreme temperature.
  \item Helps in protein synthesis and nucleic acid formation: Cytokinins are known to undergo extensive inter-conversions between the free base (nucleobase), ribosides and ribotides when experimentally supplied to tissues.
  \item Retention of chlorophyll molecule (Richmond Lang effect).
\end{itemize}

\end{document}